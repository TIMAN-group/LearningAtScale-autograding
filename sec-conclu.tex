\section{Conclusions and Future Work}

Automated grading of complex assignments is necessary for scaling up learning without compromising 
effectiveness of learning. In this paper, we have experimentally shown the feasibility of using
supervised learning techniques for automated grading of complex
assignments (e.g., medical case assignments) under certain conditions.
First, the instructor must be able to manually label a number of the
assignments to serve as a training set from which to learn a predictive
function. Second, care must be taken to craft features that can represent
student submissions such that they can be separated from one another by
the learning algorithm. If both of these assumptions are satisfied (as
they are for our ``quality'' rubric dimension), an ordinal regression
method can be applied to the data with results that consistently outperform
the majority-label baseline in terms of mean absolute error.

While the problem is not completely solved, this method shows promise. We
believe that with more work, complex outline-based assignments such as the
medical case assignments we study here can be autograded sufficiently well
across many different rubric dimensions. We expect our findings to be
applied to the veterinary medicine class from which this data was obtained.
The automation provided by our method will enable faster feedback to
students than was previously possible, and we will continue to work on
reducing the method's error rate on the other rubric dimensions.

We also proposed a general framework for the development of three complementary types of 
representative
features for student submissions (i.e., token features, similarity features, and selection features) ---while we applied these features to our
specific task of medical case analysis grading, these feature types (and
generation framework) are general and should apply to the grading of any
complex assignment.

Moving forward, we should consider new feature instantiations that follow
(or extend upon) our feature generation framework in order to better
capture the patterns we require for grading different dimensions such as
``clarity''. One example might be spelling/grammar mistakes, which could in
principle be extracted by taking a token-based view. Novel features not
explored here could also extend to evaluating the quality of references
within the assignment.

We also believe that an active learning approach should be effective for
automating the training set selection of our method, which is currently a
weakness of the proposed solution. We would like to address the
effectiveness of an active learning-based automatic grading approach.
Another limitation of our study is that the data set used for evaluation is small. 
This is partly due to the fact that such complex assignments currently can only be graded
by human graders. In the future, we hope to deploy our automated grading tools
to help scale up such courses to enable more students to participate, which in turn, would
help collecting more data for experiments. 

Another direction that remains unexplored is feedback: how could such a
system give more detailed feedback to students beyond just their ordinal
rating along a rubric dimension? Currently, peer grading approaches have
an advantage in this sense, as your peers can suggest to you corrections
or point out specific mistakes that you made. It is worth investigating
whether or not we can generate ``explanatory reports'' of grading results
when using a supervised learning approach such as this.
