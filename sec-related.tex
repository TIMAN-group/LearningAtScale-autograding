\section{Related Work}

To the best of our knowledge, no previous work has studied how to automatically grade
a complex assignment such as a medical case assessment. However, our work
is related to multiple lines of existing work, which we briefly review below. 

Automated grading has been explored mostly for constrained question
types where the correct answer has a certain, well known form. Programming
assignments, for example, have long been a target for automatic
grading~\cite{Forsythe:1965:CACM, Helmick:2007:ITICSE} as their very medium
can easily be leveraged for providing ``yes'' or ``no'' feedback with
respect to programmatic correctness. For specific assignment types, more
sophisticated techniques like edit-distance of canonical representations
has been explored~\cite{Alur:2013:IJCAI}. Recent efforts have focused on
providing feedback to students about their programs by leveraging
structural similarities in the code itself to allow feedback to be provided
to many assignments at once that share particular
features~\cite{Nguyen:2014:WWW, Piech:2015:ICML}.

In this vein, clustering-based techniques have been applied to tools
designed to help instructors manually grade short-answer MOOC assignments
at scale by allowing them to assign grades to entire clusters of students
at once~\cite{Brooks:2014:Powergrading}. Hierarchical clustering methods
were applied in this work to allow the instructor to ``drill down'' as far
as he/she would like to assign grades and feedback to students. Their
method, PowerGrading, can be regarded as optimizing the collaboration of
humans and machines heuristically, but the approach does not take advantage
of supervised learning from graded examples of instructors, which we
explore. Furthermore, if a cluster is poorly formed, the grading error can
be serious no matter how an instructor optimizes the grade assignment to a
cluster.

Mitros et.\ al.~\cite{Mitros:2013:MITLINC} give a brief overview of
different strategies for grading and proposed a heuristic workflow to
optimize the collaboration of assessors in consideration of different costs
associated with different graders. However, it does not address the
question of how to optimize the recommendation of assignments for graders
to grade in order to maximize the effectiveness of the machine learning
component of their framework, something we seek to address in this paper.
We could deploy our technology in their framework by modifying the
threshold strategy (e.g., for cutoffs on a ranked list). Both of these
methods~\cite{Brooks:2014:Powergrading, Mitros:2013:MITLINC} only explored
the short-answer question space, leaving semi-automated grading of more
complex assignment outputs (like the outline-form case assessments we study
here) unexplored.

Another approach would be to attempt to predict the grades explicitly. One
branch of work in this direction based on information extraction techniques
focuses on extracting fixed patterns from the answer: if the pattern (or
set of patterns) is matched, the question is correct, otherwise it receives
partial or no credit. Many methods require the manual construction of these
patterns~\cite{Mitchell:2002:ICAA, Leacock:2003:CatH}, while others attempt to
learn them from large training datasets~\cite{Pulman:2005:EdAppsNLP}. In either
case, the methods require strong supervision support to be effective.  Other
works take an unsupervised text-similarity approach and compare the student
answers with a gold standard answer using a wide variety of similarity
functions~\cite{Mohler:2009:EACL}.

Grading of long-form student answers has also been
explored~\cite{Balfour:2013}. In
\textsc{CarmelTC}~\cite{Rose:2003:HLT-NAACL-EDUC} a combination of topic
modeling and text classification approaches are taken to score student
essays. The system attempts to determine which ``key components'' have been
mentioned in each essay and uses this information to suggest to students
what components they may be missing. Approaches that purely use document
similarity metrics~\cite{Duwairi:2006:CHB} or purely supervised
classifiers~\cite{Larkey:1998:SIGIR} have been used for grading as well,
but the rubrics are not as complex as those required for medical case
assessment.

The task of predicting categorical labels with an implicit ranking (ordinal
variables) is often solved via ordinal regression
methods~\cite{McCullagh:1980}. These types of variables (and this
regression method) is very common in studies in the social sciences where
survey responses are often measured on scales such as (strongly disagree,
disagree, neutral, agree, strongly agree) which are called Likert
scales. Our work in predicting the grade labels for different dimensions of
an assignment's rubric can be seen as yet another application of ordinal
regression to the many applications already explored. Using machine
learning for optimizing ranking has been extensively in information
retrieval~\cite{Liu:2009}; our work explores an interesting novel
application of online active learning to automated grading where we are
interested in minimizing the training sample to label while maximizing the
ranking accuracy over a finite number of known test cases.  
