\section{Related Work}

To the best of our knowledge, no previous work has studied how to automatically grade
a complex assignment such as a medical case assessment. However, our work
is related to multiple lines of existing work, which we briefly review below. 

Automated grading has been explored mostly for constrained question
types where the correct answer has a certain, well known form. Programming
assignments, for example, have long been a target for automatic
grading~\cite{Forsythe:1965:CACM, Hext:1969:CACM, Helmick:2007:ITICSE,
Martin:1973:SIGCSE} as their very medium can easily be leveraged for
providing ``yes'' or ``no'' feedback with respect to programmatic
correctness.

Many techniques for short-answer grading have been explored. Recently,
clustering-based techniques have been applied to tools designed to help
instructors manually grade MOOC assignments at scale by allowing them to
assign grades to entire clusters of students at
once~\cite{Brooks:2014:Powergrading}. Hierarchical clustering methods were
applied in this work to allow the instructor to ``drill down'' as far as
he/she would like to assign grades and feedback to students. This method
is general and unsupervised, but cannot provide the instructor with a
suggestion for the grade or feedback so this work must be done manually.

Another approach would be to attempt to predict the grades explicitly. One
branch of work in this direction based on information extraction techniques
focuses on extracting fixed patterns from the answer: if the pattern (or
set of patterns) is matched, the question is correct, otherwise it receives
partial or no credit. Many methods require the manual construction of these
patterns~\cite{Mitchell:2002:ICAA, Leacock:2003:CatH}, while others attempt to
learn them from large training datasets~\cite{Pulman:2005:EdAppsNLP}. In either
case, the methods require strong supervision support to be effective.  Other
works take an unsupervised text-similarity approach and compare the student
answers with a gold standard answer using a wide variety of similarity
functions~\cite{Mohler:2009:ACL}.

Grading of long-form student answers has also been
explored~\cite{Balfour:2013, Chen:2014:IRRODL}. In
\textsc{CarmelTC}~\cite{Rose:2003:HLT-NAACL-EDUC} a combination of topic
modeling and text classification approaches are taken to score student
essays. The system attempts to determine which ``key components'' have been
mentioned in each essay and uses this information to suggest to students
what components they may be missing. Approaches that purely use document
similarity metrics~\cite{Duwairi:2006:CHB} or purely supervised
classifiers~\cite{Larkey:1998:SIGIR} have been used for grading as well.

The task of predicting categorical labels with an implicit ranking (ordinal
variables) is often solved via ordinal regression
methods~\cite{McCullagh:1980}. These types of variables (and this
regression method) is very common in studies in the social sciences where
survey responses are often measured on scales such as (strongly disagree,
disagree, neutral, agree, strongly agree) which are called Likert
scales~\cite{Likert:1932}. Our work in predicting the grade labels for
different dimensions of an assignment's rubric can be seen as yet another
application of ordinal regression to the many applications already
explored.
