\documentclass{sigchi}

% Use this command to override the default ACM copyright statement (e.g. for preprints).
% Consult the conference website for the camera-ready copyright statement.


%% EXAMPLE BEGIN -- HOW TO OVERRIDE THE DEFAULT COPYRIGHT STRIP -- (July 22, 2013 - Paul Baumann)
% \toappear{Permission to make digital or hard copies of all or part of this work for personal or classroom use is 	granted without fee provided that copies are not made or distributed for profit or commercial advantage and that copies bear this notice and the full citation on the first page. Copyrights for components of this work owned by others than ACM must be honored. Abstracting with credit is permitted. To copy otherwise, or republish, to post on servers or to redistribute to lists, requires prior specific permission and/or a fee. Request permissions from permissions@acm.org. \\
% {\emph{CHI'14}}, April 26--May 1, 2014, Toronto, Canada. \\
% Copyright \copyright~2014 ACM ISBN/14/04...\$15.00. \\
% DOI string from ACM form confirmation}
%% EXAMPLE END -- HOW TO OVERRIDE THE DEFAULT COPYRIGHT STRIP -- (July 22, 2013 - Paul Baumann)


% Arabic page numbers for submission.
% Remove this line to eliminate page numbers for the camera ready copy
% \pagenumbering{arabic}


% Load basic packages
\usepackage{balance}  % to better equalize the last page
\usepackage{graphics} % for EPS, load graphicx instead
\usepackage{times}    % comment if you want LaTeX's default font
\usepackage{url}      % llt: nicely formatted URLs
\usepackage{cite}

% llt: Define a global style for URLs, rather that the default one
\makeatletter
\def\url@leostyle{%
  \@ifundefined{selectfont}{\def\UrlFont{\sf}}{\def\UrlFont{\small\bf\ttfamily}}}
\makeatother
\urlstyle{leo}


% To make various LaTeX processors do the right thing with page size.
\def\pprw{8.5in}
\def\pprh{11in}
\special{papersize=\pprw,\pprh}
\setlength{\paperwidth}{\pprw}
\setlength{\paperheight}{\pprh}
\setlength{\pdfpagewidth}{\pprw}
\setlength{\pdfpageheight}{\pprh}

% Make sure hyperref comes last of your loaded packages,
% to give it a fighting chance of not being over-written,
% since its job is to redefine many LaTeX commands.
\usepackage[pdftex]{hyperref}
\hypersetup{
pdftitle={SIGCHI Conference Proceedings Format},
pdfauthor={LaTeX},
pdfkeywords={SIGCHI, proceedings, archival format},
bookmarksnumbered,
pdfstartview={FitH},
colorlinks,
citecolor=black,
filecolor=black,
linkcolor=black,
urlcolor=black,
breaklinks=true,
}

% create a shortcut to typeset table headings
\newcommand\tabhead[1]{\small\textbf{#1}}


% End of preamble. Here it comes the document.
\begin{document}

\title{An Exploration of Automated Grading of Complex Assignments}

\numberofauthors{3}
\author{
  \alignauthor 1st Author Name\\
    \affaddr{Affiliation}\\
    \affaddr{Address}\\
    \email{e-mail address}\\
    \affaddr{Optional phone number}
  \alignauthor 2nd Author Name\\
    \affaddr{Affiliation}\\
    \affaddr{Address}\\
    \email{e-mail address}\\
    \affaddr{Optional phone number}
  \alignauthor 3rd Author Name\\
    \affaddr{Affiliation}\\
    \affaddr{Address}\\
    \email{e-mail address}\\
    \affaddr{Optional phone number}
}

\maketitle

\begin{abstract}
Automated grading is essential for scaling up learning. In this paper, we
conduct the first study of how to automate grading of a complex assignment
using a medical case assessment as a test case. We propose to solve this problem
using a supervised learning approach (ordinal regression), and propose a general
methodology for designing three complementary types of feature representations of such complex
assignments, including token features, similarity features, and selection features. 
%  and propose the use of an ordinal regression approach using that
Experiment results show that it is feasible to automate
grading of such assignments provided that the instructor can grade a number of
examples. However, careful feature engineering tailored towards specific
grading criteria is critical for the feasibility of autograding. 
\end{abstract}

\keywords{
  Automatic grading; ordinal regression; supervised learning; text mining
}

\category{H.2.8}{Database Applications}{Data mining}

\def\ignore#1{}
\section{Introduction}

Information Technologies have been transforming education dramatically
recently, leading to the rapid growth of the Massive Open Online Courses
(MOOCs), which have not only made education more affordable and scalable,
but also have huge potential to enable more effective personalized
learning.  A key component technology that enabled the success of MOOCs is
the automatic grading capability of a MOOC system. Unfortunately, the
current technology for automatic grading is mostly limited to multi-choice
questions, short answers~\cite{Brooks:2014:Powergrading,
Leacock:2003:CatH, Mitchell:2002:ICAA, Pulman:2005:EdAppsNLP,
Mohler:2009:EACL}, and simple essay scoring~\cite{Chen:2014:IRRODL,
Balfour:2013}, which makes it impossible for the current MOOCs to provide
sophisticated assignments that are needed for teaching complex concepts or
skills (e.g., critical thinking skills) since they cannot be easily graded
in a scalable way. A solution currently adopted to bypass this difficulty
is to use the calibrated peer review~\cite{Balfour:2013, Chen:2014:IRRODL,
Sandeen:2013, Suen:2014}.  However, there are systematic problems with this
approach: discrepancy between peer and instructor ratings, variation in
ratings over time by the same peer rater, inconsistency across exercises
for rating two works of similar quality, differences in rater stringency,
and random fluctuation of ratings of the same work under varied
conditions~\cite{Suen:2014}.  Preliminary data from a recent attempt to use
this technique with veterinary students has also shown that peer reviews
have a distinct positive bias (vide infra) relative to an expert instructor
analysis~\cite{Ferguson:2014}.  Thus it is important to develop more
powerful automatic grading technology that can be applied to more
sophisticated exercises than those provided by the current MOOCs, which are
necessary in many education scenarios.

To automate grading of such a complex assignment, a natural idea is to use
supervised machine learning to learn from the graded examples for
automatically assigning grades to those ungraded ones. As in other machine
learning applications, the general idea here is that if we can extract
those features from the assignments that can indicate the quality of an
assignment, a machine learning program would be able to pick up the
patterns of the features that can distinguish high-quality work from
low-quality work from a sample of graded assignments (i.e., ``training
data''), thus potentially assigning a grade automatically to an ungraded
assignment.

Although this idea is natural and appealing, however, there are many challenges and
questions that we must address before we can effectively deploy such a
technology in a real education environment, and a main goal of this paper
is to take a first step toward systematically addressing these questions,
including specifically the following questions:

\begin{enumerate}
\item {\bf Feasibility:} How feasible is it to use machine learning to automate grading of a complex assignment? 
What general features can we extract from assignments for automated
 grading? How effective are the state of the art machine learning approaches
 for automated grading? Are they sufficiently effective to be immediately
 useful in practice?

\item {\bf Problem Formulation and Evaluation:}  How should we formulate the grading problem as a machine learning
 problem? There are at least two options. One is to frame it as a
 classification problem with the goal of classifying an assignment into one
 of the finite number of pre-defined grade levels based on the rubrics. The
 other is to frame it as a ranking problem where the goal is to rank the
 assignments based on the quality without necessarily assigning a specific
 level of grade, and human graders can then go through the ranked list to
 segment the assignments into different grade levels. How should we design evaluation metrics
to measure the quality of the results of automated grading? 

\item {\bf Integration of Automated Grading and Human Grading:} How exactly should such an automated grader be integrated with
 instructor or TA grading? A more general question is: how can we optimize
 the collaboration of an imperfect automated grader with more reliable
 human graders? Intuitively, the optimization depends on a trade-off between
 the quality/reliability of grading and the amount of human effort
 required. But given an expected amount of human effort, what is the best
 way to have the automated grader to assist human in grading, and to have
 humans to help train the machine learning-based automated grader?

\end{enumerate}

We will study these questions using a particular type of complex
assignments that require sophisticated critical thinking skills, i.e.,
medical case assessment. This kind of assignments are very important for
medical professional education. Thus by studying how to automate grading
for medical assessment assignments, we can potentially enable medical
professional education to scale up, a much needed effort. 
\ignore{
Traditional
veterinary curricula have usually consisted of didactic lectures for as
long as 3 years before entry into the clinic. In veterinary medicine, this
approach has been shown to result in an actual decline in critical thinking
skills~\cite{Herron:1992}. Therefore, to encourage self-awareness and
lifelong learning, didactic instruction should be supplemented by practice
with the process of information gathering and critical clinical thinking
associated with diagnosis of disease and selection of treatment. In her
book on critical thinking in clinical practice~\cite{Gambrill:2005},
Gambrill notes that evidence-based practice is rooted in a willingness to
recognize the intrinsic uncertainty of clinical decision-making.
%Not all clinicians are willing to acknowledge
%this uncertainty in their mission to help patients. This deficiency in
%training in critical clinical thinking is thus not unique to our veterinary
%curricula; in fact,
Furthermore, 
}
Not teaching clinicians about clinical uncertainty has
been referred to as ``the greatest deficiency of medical education
throughout the twentieth century''~\cite{Djulbegovic:2004, Gambrill:2005}.
\ignore{
Evidence-based medicine (EBM) requires the skillset to develop answerable
questions relevant to a case, and to answer these questions with an honest
and open appraisal of research findings~\cite{Braddock:1999}.
Learning by doing is emphasized in EBM and evaluation of case
studies provides such practice. Instructors need to be prepared to provide
structure to the process together with timely feedback during practice in
order to be effective~\cite{Gambrill:2005}. }
However, implementing an instruction plan with an online education system at large scale
to teach clinical uncertainty in decision making raises many significant challenges that must be solved, particularly
challenges in automatic evaluation of the case studies completed by the
students, which we address in this paper by leveraging information
retrieval and machine learning techniques.

To study the feasibility questions, we propose a general methodology for designing three
complementary types of feature representations of such complex assignments,
including token features, similarity features, and selection features, and experiment
with these features using ordinal regression for predicting the grade levels in multiple
dimensions of rubrics.  The
token features are based on the term tokens extracted from an assignment
and they offer the most general and robust representation. The similarity
features are to capture the similarity between an assignment and the
solution provided by an instructor; the intuition is that the higher the
similarity is, the higher the grade should be.  Finally, the selection
features are to quantify the accuracy of the selection of relevant parts in
a case description based on how well the selected parts match the
solutions.  While it is generally beneficial to manually design
assignment-specific features, such features cannot be generalized to work
on other assignments, in this paper, we focus on studying {\em general}
features that can be {\em automatically computed} on any semi-structured
complex assignments, and aim at understanding their effectiveness.

A practical challenge in studying our problem is the lack of a large set of
graded assignments which is needed both for training a machine learning
program and for validating the results of automated grading. This is partly
due to the fact that grading such assignments takes much human effort, the
very reason why we need to study automated grading for such assignments. In
our experiments, we used a data set of 107 student submissions for one
medical case assessment assignment that are available to us. While the data
set is small, we were able to observe statistically significant differences
in our experiments, thus it still allows us to draw meaningful conclusions
about different approaches to automated grading. 

Our study with this data set shows that it is feasible to automate grading of 
a complex assignment such as medical case assessment using standard
 machine learning approaches and the proposed three kinds of general features 
provided the instructor can grade a small number of examples, but 
the grading accuracy on different rubric categories vary substantially. 

The results of our feasibility study reveal that there is a great deal of 
variation in the grades given by instructorrs due to the inevitable subjectivity of the rubrics. 
This suggests that  it might be less effort and more reliable for an instructor to make pairwise judgments between
a pair of assignments  as opposed to assigning an exact numerical or letter grade. Working 
on such pairwise preference judgments also makes it easy to integrate
 non-expert judgments (such as peer grading) that might not be reliable in the exact grades assigned but may include relatively reliable preference judgments. Moreover, working on pairwise preferences naturally ``eliminate" the need for normalizing numerical grades which might be biased 
(e.g., some graders may be overly generous).  
%as well  imagine having peer graders provide the higher-accuracy, bias-free "which is better" judgments rather than having to assign their peers a specific grade. These judgments can then be used as supervision, perhaps with different instance weights than if they were coming from an instructor.
Given that we will attempt to obtain pairwise preferences as training examples, it follows that we should frame the problem of automated grading as to rank the ungraded assignments, as opposed to predict the exact grade of an assignment. The ranking would be in descending order of quality (in any rubric dimension or overall quality with consideration of multiple dimensions), and a human grader can  then easily segment the list into any desired grade levels.  In comparison with predicting exact grades, such a ranking formulation also offers a natural way to engage humans in validating and finalizing the grades. 
%In comparison with directly assigning a grade level to each assignment using machine learning, such a ranking
%strategy has several advantages: .... {\bf [list the benefits or our motivation  here]}. 
For evaluation, althoough retrieval measures such as Mean Average Precision (MAP) or normalized Discounted Cumulative Gain (nDCG) are commonly used for evaluating a ranked list, we suggest that the Normalized Distance-based Performance Measure (NDPM) is a better measure for our ranking problem since it can better handle many ties that inevitably exist in our case. 

In practice, an automated grader must be integrated with a human grader so as to minimize the overall effort of the human grader while ensuring a certain level of grading accuracy. There is an inherent tradeoff here since to increase the grading accuracy, we would like to have as many training examples (i.e., manually graded assignments) as possible, which, however, would incur more human effort. To optimize human-machine collaboration and enable a flexible tradeoff between human effort and grading accuracy, we propose the following sequential
training process based on active machine learning: 1) human grader first grades a small number of assignments as the initial training set; 2) the machine would learn from the intial set, and identify the next ``best" example (i.e., assignment) to label and present it for human to grade where ``best" means that the example is most valuable to help the automated grader improve its accuracy; 3) human would grade the nominated example to increase the size of the training set by one; 4) machine would learn from the augmented training set and repeatedly present a new example for human to grade until it reaches a desired level of accuracy, at which point, the process stops and the human grader would segment the final ranked list to generate grades for all the assignments. Our experiment results show that such an active learning process is much more effective than batch training. 

As the first study of this new problem, our study is inevitably limited by
the size of the data set due to the unavailability of larger data sets; as the automated grading technology matures for
such assignments, we anticipate to have much larger data sets available to
further verify our observations and conclusions.



\section{Related Work}

To the best of our knowledge, no previous work has studied how to automatically grade
a complex assignment such as a medical case assessment. However, our work
is related to multiple lines of existing work, which we briefly review below. 

Automated grading has been explored mostly for constrained question
types where the correct answer has a certain, well known form. Programming
assignments, for example, have long been a target for automatic
grading~\cite{Forsythe:1965:CACM, Hext:1969:CACM, Helmick:2007:ITICSE,
Martin:1973:SIGCSE} as their very medium can easily be leveraged for
providing ``yes'' or ``no'' feedback with respect to programmatic
correctness.

Many techniques for short-answer grading have been explored. Recently,
clustering-based techniques have been applied to tools designed to help
instructors manually grade MOOC assignments at scale by allowing them to
assign grades to entire clusters of students at
once~\cite{Brooks:2014:Powergrading}. Hierarchical clustering methods were
applied in this work to allow the instructor to ``drill down'' as far as
he/she would like to assign grades and feedback to students. This method
is general and unsupervised, but cannot provide the instructor with a
suggestion for the grade or feedback so this work must be done manually.

Another approach would be to attempt to predict the grades explicitly. One
branch of work in this direction based on information extraction techniques
focuses on extracting fixed patterns from the answer: if the pattern (or
set of patterns) is matched, the question is correct, otherwise it receives
partial or no credit. Many methods require the manual construction of these
patterns~\cite{Mitchell:2002:ICAA, Leacock:2003:CatH}, while others attempt to
learn them from large training datasets~\cite{Pulman:2005:EdAppsNLP}. In either
case, the methods require strong supervision support to be effective.  Other
works take an unsupervised text-similarity approach and compare the student
answers with a gold standard answer using a wide variety of similarity
functions~\cite{Mohler:2009:EACL}.

Grading of long-form student answers has also been
explored~\cite{Balfour:2013, Chen:2014:IRRODL}. In
\textsc{CarmelTC}~\cite{Rose:2003:HLT-NAACL-EDUC} a combination of topic
modeling and text classification approaches are taken to score student
essays. The system attempts to determine which ``key components'' have been
mentioned in each essay and uses this information to suggest to students
what components they may be missing. Approaches that purely use document
similarity metrics~\cite{Duwairi:2006:CHB} or purely supervised
classifiers~\cite{Larkey:1998:SIGIR} have been used for grading as well.

The task of predicting categorical labels with an implicit ranking (ordinal
variables) is often solved via ordinal regression
methods~\cite{McCullagh:1980}. These types of variables (and this
regression method) is very common in studies in the social sciences where
survey responses are often measured on scales such as (strongly disagree,
disagree, neutral, agree, strongly agree) which are called Likert
scales~\cite{Likert:1932}. Our work in predicting the grade labels for
different dimensions of an assignment's rubric can be seen as yet another
application of ordinal regression to the many applications already
explored.


\section{Medical Case Assignment}

Complex assignments inevitably vary across courses. As a first step toward
studying how to automate grading of such assignments, we  use a medical
case assignment in the veterinary medicine domain for our study. At a high
level, such an assignment represents a typical type of analysis assignment
where the students are given a case description with both an unstructured
text description as well as some structured data (e.g., lab test results),
and are asked to perform an analysis of the case. The analysis generally
involves 1) selecting relevant content from the case description, which can
be selected from both the text part and the structured data, 2) answering
questions with short textual answers, and 3) writing assessments in
natural language text.

More specifically, the case exercises were developed using
WhenKnowingMatters (WKM) web-based case formulation
software{\footnote{available at \url{http://www.whenknowingmatters.com}.}
which facilitates development and exchange of text-based cases while
allowing students to objectify their observations from a case and
manipulate them in an outline format around a suggested scaffold provided
by the instructor. The student's analysis is then rendered into a
structured text format to facilitate automatic grading.

\begin{figure*}[ht]
\includegraphics[scale=0.5]{case-desc-small.png}
\includegraphics[scale=0.6]{student-work-small.png}
\caption{An example of case description (left) and student assessment (right)}
\label{fig:example}
\end{figure*}

Due to the lack of automated grading tools, the assignments are currently
graded manually. An assessment rubric designed prior to instruction was
used by the instructor to evaluate student performance on a subjective,
5-point scale (listed here in increasing order): novice, beginner,
competent, proficient, and expert. Rubric categories were related to
elements of critical thinking and communication:
\begin{enumerate}
\item {\bf Questions:} Developing relevant refining (or clarifying)
 questions to answer based upon an honest assessment of current knowledge
 base.
\item {\bf Answers:} Approach to seeking answers to developed
 questions; literature search, etc.
\item {\bf Quality:} Judgment of quality of information; awareness and
 application of standards of a discipline, bias detection including
 appropriate humility to detect one’s own potential bias, application of
 statistical concepts.
\item {\bf Analysis:} Analysis of an argument.
\item  {\bf Clarity:} Clarity and communication of thought; conciseness,
 grammar, spelling, elocution.
\item {\bf Application:} Application and understanding of appropriate
 disciplinary content.
\end{enumerate}

The instructor also created a ``gold standard" assessment for the
assignment case, which is available for the automated grading tool to use.

Figure~\ref{fig:example} shows a sample case description and a typical
student answer. In the case description, the student can see a text
description of the case and a number of lab test results in the form of
structured data. The student assessment is seen to be a semi-structured
text with indented structures based on a scaffold provided by the
instructor. Multiple tags indicate different kinds of answers, including,
e.g., selected content from the original case description, selected lab
tests, and text input by the student reflecting his/her assessment.

Because of the complexity, automated grading of such an assignment is very
challenging. First, because of variations across different assignments, it
is almost impossible to learn from the grading results of one assignment to
automate grading of another, even though such an ``inter-assignment"
automated grading is ideal.  We thus focus on a more realistic setting of
attempting to automate the grading after the instructor has already graded
some assignments, which we may refer to as ``intra-assignment" automated
grading, which, strictly speaking, is actually semi-automatic grading.  Our
goal is thus to study whether and how we can leverage machine learning to
learn from the graded assignments to automatically predict the grades for
the rest of the ungraded assignments.


\section{Feasibility of Automated Gradiing}

In this section, we discuss and study the feasibility of using machine learning methods, particularly supervised learning,  for automating the grading of complex assignments. We first present the general idea of supervised learning, then propose a general methodology for designing three complementary types of features for representing assignments, which are needed for supervised learning, and finally present experiment results.

\ignore{
\subsection{Unsupervised Learning}
%
Our first thought is to use unsupervised learning since it does not require any training data (i.e., graded assignments as examples).  In unsupervised learning, a model is built to describe some latent
structure of the data without considering any label information. Typical
examples of unsupervised methods include clustering and topic modeling.
Unsupervised methods have been applied to automated grading typically in
the form of clustering---the ``latent structure'' to be extracted are
groups of student submissions, which can then be collectively assigned a
single grade. Hierarchical clustering methods allow an instructor to
``drill down'' by dividing a single cluster into several smaller ones that
somehow capture differences within the same group. This method was
exploited to great effect by the power grading
project~\cite{Brooks:2014:Powergrading} to allow instructors to spend
significantly less time grading short answer questions.
%
However, this does not address the need to evaluate students along
different dimensions as occurs in rubric grading for complex assignments.  Because the method does
not consider the labels in learning the latent structure, the structure it
finds will tend to be general with no principled way of tuning it to
better describe the desired label outcomes. Furthermore, this strategy
also provides no guidance for the instructor in assigning the actual grade
value itself---while it aids in digesting the patterns that occur in the
data, the instructor is still on his/her own in deciphering what the grade
should be based on these patterns. With simple questions with more or less
one correct answer, this is not a problem---but when we attempt to address
more complex assignments that involve critical thinking and principled
analysis (for which there are many ``correct'' answers) this becomes much
more difficult and time consuming.
}

\subsection{Supervised Learning}

In supervised learning, a model is built to predict the outcome (or label)
of a new data example based on previous examples it has seen before (called the
training data). Thus a natural way to use supervised learning for grading
is to have a human (e.g., instructor) to grade a set of assignments to be used
as training data to learn a model to predict the grade of each ungraded
assignment. A critical component of this infrastructure is the
decomposition of examples into feature vectors---this decomposition enables
the use of algorithms for learning functions from these vectors to the
output labels desired. Typically, these feature vectors are either binary
or real-valued, and are often (but not always) in a high-dimensional space.
The performance of the learned function is directly tied to the features
used in the vector representation of the examples---poor features result
in low predictive capability due to the algorithm being unable to find
meaningful patterns in the examples. As such, these features are a
critical component of any supervised learning approach. With a properly
defined set of features that are capable of capturing the salient patterns
in the training examples, the task can be given to any of a number of
state-of-the-art algorithms for learning appropriate predictive functions
that can be applied to yet-unseen data (the test data). Another factor
affecting the accuracy of prediction is the number of training examples, with
more training examples leading to higher accuracy. However, since
creating training examples generally requires manual work, we tend
to have only a limited number of training examples to work with.
How to define general features that we can automatically compute
based on a complex assignment and how to minimize manual effort
in creating training examples are two major questions that we study in this paper.

\ignore{This scenario is applicable to an automated grading setup in which the
instructor labels a certain number of student assignments with a grade
which are then fed in to the algorithm to learn a predictive function that,
when applied to the unlabeled examples, provides a grade based on the
patterns extracted from the manually graded examples.
%
This supervised framework would allow instructors to learn a separate
predictor for each dimension in a rubric by simply changing the labels of
the training examples to reflect a different rubric dimension during
training. We will explore supervised learning approaches in this paper to
address our need for predicting different rubric dimensions instead of a
single overall grade. In such a setting, the grading problem has been
reduced to defining a set of features that characterize a student's
assignment along multiple dimensions.}

\subsubsection{Defining Features of a Student Assignment}

The performance of a supervised learning approach is highly dependent on
the effectiveness of the features fed into the learning program. Thus a main technical
challenge we need to solve is how to design effective features for representing
an assignment.

To address this challenge, we propose a general framework for defining features for complex
assignments such as the one we explore in this paper. The features we
propose are general in nature and thus should be applicable to any
assignment that is presented in a text-based, semi-structured response
form. We describe a set of feature classes and evaluate the performance of
these features on an example autograding task to evaluate their predictive
capacity. Our framework consists first of constructing a ``view'' of an
assignment and then defining features based on this view. The view chosen
for the assignment is critical in that it changes the way we may naturally
describe it and thus leads to the definition of distinct classes of
features distinguished by the view taken to derive them. We will explore
features by progressively taking views that make stronger assignment design
assumptions: while the features are still general, each view progressively
narrows the space of possible student response types.

The first class of features, which we call \textbf{token features}, are
generated by taking a view of the student response consistent with the
traditional ``bag of words'' approach common in information retrieval
contexts~\cite{Manning:2008}. In this view a document is decomposed into
a vector of count data that indicates the frequency of words within the
document. Two features are thus natural. The first type of feature would
indicate the number of occurrences of a given word in a student submission
(and is thus real-valued), and the second would indicate the presence or
absence of a word (and is thus binary-valued). These features would both
create a high-dimensional representation of the student submission, and are
motivated by an attempt to capture the difference in vocabulary between
assignments. This is often enough to capture whether the correct ideas are
mentioned without requiring extensive computation (features from this class
are trivial to compute for every student submission), though more discriminative
units such as n-grams (a sequence of $n$ words) may also be easily used
to replace single words if necessary.  Document
classification techniques typically operate in this kind of space.

The second class of features, which we call \textbf{similarity features},
are generated by characterizing a student submission by the ``distance''
from a gold standard (e.g., an assignment submission generated by the
instructor). With this view, features can be derived that strongly utilize
the structure of the assignment (e.g., how closely does the outline
structure of the student assignment match the outline structure of the
instructor assignment?)\ as well as features that loosely utilize or
completely ignore the structure of the assignment. Examples of features
that loosely utilize the assignment structure would be the similarity of
certain outline bullet types with the gold standard bullet types of the
same category. A bullet type in our examples could be ``observation''
(indicating something selected from the assignment text directly) or
``analysis'' (indicating original thoughts from the student). These
features require the assignment to be structured in such a way that this
information is easily extracted, but do not look so closely at the overall
structure of the outline itself. Ignoring the structure of the assignment,
features can be generated that indicate the overall similarity with the
gold standard. Document clustering techniques typically operate in this
kind of space, as well as retrieval functions in search
systems~\cite{Manning:2008}.

The third class of features, which we call \textbf{selection features},
are generated by measuring concrete statistics about the selection of
bullet points compared to a gold standard. In some sense, these are similar
to the similarity features, but they differ in that they make a stronger
assumption about the assignment structure---namely, that students are
producing the exact same text that should occur in a similar section of the
gold standard. Examples of selection features would be precision (what
fraction of the bullets selected by the student also appear in the gold
standard?)\ and recall (what fraction of bullets selected in the gold
standard were also selected by the student?)~\cite{Manning:2008}.

\ignore{
\subsection{Active Learning}
One critical problem in the supervised learning setting is the selection
of a training set. If a non-representative training set is selected, the
algorithm has no opportunity to learn the features that distinguish the
excellent submissions from the ordinary submissions. Unfortunately, the
traditional supervised learning setting offers no principled mechanism for
picking which training set to use---it just assumes one exists a priori.
%
Active learning methods~\cite{Settles:2012} bridge this gap by providing a
mechanism for selecting relevant training examples designed to maximally
improve the performance of an existing model. This setting is very relevant
for an autograding setup, where the system should ideally ask the
instructor to grade a \emph{specific} set of examples, rather than forcing
the instructor to find good representative examples on his/her own. This
process can be iterative: the system can learn from the first batch of
examples graded by the instructor, and then request him/her to grade a
second batch, which is used to incrementally improve the learned model.
This should, in principle, reduce the amount of time an instructor would
have to spend grading to obtain a certain performance threshold for the
grade predictor.
%
\subsection{Combined Methods for Complete Grading Support}
%
A comprehensive system can then be designed that leverages all three of
these perspectives: unsupervised, (semi-) supervised, and active learning.
We can first apply an unsupervised learning technique to group the student
assignment data into rough categories. Then, a set of student submissions
can be sampled from these distinct categories, resulting in a collection of
submissions that are in some sense different from one another---these
examples are then labeled by the instructor and used as the starting point
for an active learning method: we can learn a supervised classifier using
the now labeled data, use this classifier to separate the remaining
unlabeled training data, sample more documents for the instructor to label,
and continue in an iterative fashion until either a fixed evaluation
criteria is met, or a certain number of submissions have been labeled.
Such a combined method would attempt to {\bf optimize the collaboration of
human graders and the automated grading tool}  so that we can leverage the
best of each.
}

\subsubsection{Ordinal Regression for Grade Prediction}

Because of the ordinal nature of our grade labels (categorical with an
implicit ranking), it is natural to apply ordinal regression techniques to
our automated grading setup. In particular, we will utilize support vector
ordinal regression (SVOR)~\cite{Chu:2007:SVOR}, a generalization of the
popular support vector machine (SVM)~\cite{Cortes:1995:SVM} for
classification to the case of ordinal labels in the study of feasibility of
grade prediction.

\subsection{Experiment Results}

We now present the results of ordinal regression on our medical assignment
data set to assess the effectiveness of the proposed features and examine
how effective such a state of the art learning method is for solving the
grading problem.

We first explored using only the most general of our feature types---token
features---to attempt to understand the differences in grading difficulty
across our different rubric dimensions. Frequency-based token features were
extracted: we used the \textsc{MeTA}
toolkit\footnote{\url{https://meta-toolkit.org}} at version 1.1
with its default tokenizer, stemmer, and stopword list. For regression, we
used a modified version of
\textsc{LIBSVM}\footnote{\url{http://www.work.caltech.edu/~htlin/program/libsvm/}}
for ordinal regression~\cite{Li:2007:NIPS}.

In an actual grading scenario, the instructor would manually grade a
certain number of the submissions, learn the regression function from these
labeled examples, and then apply the learned model to the remaining
unlabeled examples. To simulate this, we ran the following experiments: for
each rubric dimension, we took the collection of student documents and
randomly split it into two groups (the training and test sets) each
containing 50\% of the data\footnote{We do not use something like 10-fold
cross validation due to the small size of the available labeled data to
ensure that the training and test sets can be as representative of the
actual data as possible.}. A function is learned based on the labeled
training set which is then used to label the examples in the test set. We
compute the \textbf{mean absolute error} (MAE), defined as
\[
MAE = \frac{1}{n} \sum_{i=1}^n | r(f(x_i)) - r(y_i) |
\]
where $f(x_i)$ is the predicted label of the example $x_i$, $r(\cdot)$ is
the rank of a given label, and $y_i$ is the gold standard label for the
example $x_i$. This experiment is repeated for ten different randomized
splits, and we report the average and standard deviation of the test set
MAE in Table~\ref{table:train-set-size}.

\begin{table}
    \begin{centering}
        \begin{tabular}{l|c}
            analysis
            &
            \begin{tabular}{@{}c@{}}
%                0.6245 $\pm$ 0.1089\\
                0.5642 $\pm$ 0.0733
            \end{tabular}
            \\\hline

            answers
            &
            \begin{tabular}{@{}c@{}}
%                0.5623 $\pm$ 0.0290\\
                0.5491 $\pm$ 0.0634
            \end{tabular}
            \\\hline

            application
            &
            \begin{tabular}{@{}c@{}}
%                0.3170 $\pm$ 0.0375\\
                0.3321 $\pm$ 0.0580
            \end{tabular}
            \\\hline

            clarity
            &
            \begin{tabular}{@{}c@{}}
%                0.7321 $\pm$ 0.0777\\
                0.7604 $\pm$ 0.0659
            \end{tabular}
            \\\hline

            quality
            &
            \begin{tabular}{@{}c@{}}
%                0.9736 $\pm$ 0.143\\
                0.7868 $\pm$ 0.0810
            \end{tabular}
            \\\hline

            questions
            &
            \begin{tabular}{@{}c@{}}
%                0.5094 $\pm$ 0.1439\\
                0.4321 $\pm$ 0.0640
            \end{tabular}
        \end{tabular}
        \caption{Difficulty of grading each rubric dimension, characterized
        by MAE of a SVOR model learned on 50\% of the data. 10 randomized
        experiments were run; reported is the average and standard deviation.}
        \label{table:train-set-size}
    \end{centering}
\end{table}


We can observe that the rubric labels with the least variation are the
easiest to predict (e.g., ``application'' and ``questions''), whereas
rubric dimensions with higher data variance (e.g., ``quality'') are more
difficult.

\ignore{
\subsubsection{Overfitting}
A concern with complicated models such as SVMs, especially when applied to
small datasets with high dimensionality, is that of overfitting: the
training data is essentially memorized, causing the output function to
fail to perform well on new examples due to its failure to capture
patterns that generalize~\cite{Dietterich:1995:ACMCS}. Feature selection is
a method that can combat this in which only a subset of the possible
features in a given type are used~\cite{Guyon:2003:JMLR}. In our setup, we had
2646 total token features, which we attempted to reduce by only considering
tokens that had a total collection frequency above a certain threshold $k \in
\{10, 20, 100, 200\}$ resulting in 558, 383, 144, and 61 features,
respectively.  Unfortunately, this selection method failed to reduce the
overfitting phenomena, likely due to data sparsity. It would be worth exploring
more principled feature selection in future work, as this should in principle
improve the model's generalization.}

\subsubsection{The Impact of Different Feature Types}
Moving beyond simple token features, we extracted both similarity and
selection features from our assignments and incorporated them
incrementally into our model to measure the predictive capacity of
different feature types.

Our token features were generated using the same process detailed
previously (frequency-based features extracted using the \textsc{MeTA}
toolkit). Our similarity features (compared against an instructor-generated
assignment submission) were overall similarity, similarity of only
``observation'' bullets, and similarity of only ``analysis'' bullets.
These were computed using the Okapi BM25 similarity function often used in
information retrieval as a scoring function~\cite{Manning:2008}, treating
the instructor submission as a query and the student submissions as
documents to be scored. Finally, our selection features were precision and
recall~\cite{Manning:2008} of the selected lab data in the student case
analysis when compared against the instructor's assignment.

We investigate the predictive capacity of these features by exploring the
improvement of the model when predicting our most challenging labels
(``quality'' and ``clarity''). We ran ten separate experiments with
different randomized training sets consisting of 50\% of the data when
using different feature combinations to represent the student submissions.
We again report the average and standard deviation of the MAE on the test
set across the ten runs. To further explore whether the regression method
is truly capturing patterns relevant for grading, we compare its MAE
against the MAE obtained by using a na\"ive baseline: compute the most
frequent label in the training data, and then assign this label to all
examples in the test set.  Intuitively, this is a very reasonable baseline
when comparing MAE---if the labels are normally distributed, picking the
most frequent one will ensure an absolute error of zero for the majority of
the examples---while simultaneously being unhelpful for discriminative
grading (which the regression method hopes to capture). Our results are
summarized in
Tables~\ref{table:feature-comb}~and~\ref{table:feature-comb-clar}.


We see that for the ``quality'' dimension, the model is able to
successfully learn generalizable patterns in our features to predict the
label with errors that are statistically significantly less than the
baseline method. In general, the token features dominate the performance,
but it would seem as though the similarity and selection features have
lower variability in the MAE. Again, this result suggests that there are
likely gains to be had by utilizing a more sophisticated feature selection
method to remove some of the noise introduced by extraneous token features.

However, the ``clarity'' label shows us that the problem is far from being
solved in a general sense. Here, we see that our method consistently fails
to beat the baseline method, with the winning method being seemingly
random. This indicates that the features we have selected thus far are more
tailored toward discrimination along certain dimensions of the grading
rubric than others. More work must be placed into developing features that
truly capture the ``clarity'' dimension to allow the model to extract the
patterns the instructor observes when grading along this dimension.

\begin{table}
    \begin{tabular}{r|l|l}
        & \textbf{Baseline} & \textbf{SVOR}\\\hline

        \textbf{sim} (3)
        & 0.9358 $\pm$ 0.0882
        & \textbf{0.8811 $\pm$ 0.0940}
        \\\hline

        \textbf{sim + sel} (5)
        & 0.9566 $\pm$ 0.1677
        & \textbf{0.8642 $\pm$ 0.0325}
        \\\hline

        \textbf{toks} (2646)
        & 0.9075 $\pm$ 0.0789
        & \textbf{0.7660 $\pm$ 0.0910$^\dagger$}
        \\\hline

        \textbf{all} (2651)
        & 0.9792 $\pm$ 0.1568
        & \textbf{0.7566 $\pm$ 0.0738$^\dagger$}
        \\\hline
    \end{tabular}
    \\\\
    {\footnotesize
    $\dagger$: statistically signifigant using an unpaired $t$-test with $p
    \leq 0.05$.}

    \caption{Effectiveness (in terms of MAE) of incorporating additional
    features in grade prediction for ``quality'' dimensions using SVOR
    methods compared to the mode-assigning baseline. Number of features is
    given in parenthesis.}
    \label{table:feature-comb}
\end{table}

\begin{table}
    \begin{tabular}{r|l|l}
        & \textbf{Baseline} & \textbf{SVOR}\\\hline

        \textbf{sim} (3)
        & 0.7906 $\pm$ 0.0771
        & \textbf{0.7830 $\pm$ 0.0836}
        \\\hline

        \textbf{sim + sel} (5)
        & \textbf{0.7623 $\pm$ 0.0649}
        & 0.7811 $\pm$ 0.0561
        \\\hline

        \textbf{toks} (2646)
        & 0.7528 $\pm$ 0.0550
        & \textbf{0.7415 $\pm$ 0.0597}
        \\\hline

        \textbf{all} (2651)
        & \textbf{0.7189 $\pm$ 0.0617}
        & 0.7226 $\pm$ 0.0527
        \\\hline
    \end{tabular}

    \caption{Similar experiment to Table~\ref{table:feature-comb}, but for
    ``clarity'' dimension.}
    \label{table:feature-comb-clar}
\vskip-10pt
\end{table}


What this demonstrates is that automatic grading of complex assignments is
currently feasible, but perhaps only in a limited fashion. Careful feature
generation is required, but in some cases a model can be learned to
effectively grade assignments. We suspect that significant gains in
grading performance can be obtained in other dimensions with better
features.


\section{Experiments}

In this section, we perform some experiments that focus specifically on the
(semi-) supervised methods detailed in the previous section, which we view
as the starting point towards a more comprehensive method for grading
complex assignments. We detail our empirical findings on an example
dataset to evaluate our proposed methods.

\subsection{Dataset}

Our data consist of $n = 107$ student submissions for one medical case
analysis assignment in a veterinary medicine course at XXXX institution.
Each was graded along six different rubric dimensions (abbreviations
bolded): \textbf{analysis} of argument, approach in seeking
\textbf{answers}, \textbf{application} of disciplinary content,
\textbf{clarity} of communication, judgment of \textbf{quality} of
information, and developing relevant refining \textbf{questions}. Student
submissions were given an ordinal rating on a five level scale from novice
(1) to expert (5) along each dimension---we report the mean rank and
standard deviation for each of the six labels in
Table~\ref{table:grade-stats}. The ordinal regression task is then to
learn a separate function to predict the rating for each of the six
dimensions using features extracted from the student documents.

\begin{table}
    \begin{centering}
        \begin{tabular}{l|c}
            analysis & 2.6355 $\pm$ 0.7660
            \\\hline
            answers & 3.0280 $\pm$ 0.7668
            \\\hline
            application & 2.8692 $\pm$ 0.5651
            \\\hline
            clarity & 3.3832 $\pm$ 0.9437
            \\\hline
            quality & 3.1121 $\pm$ 0.9795
            \\\hline
            questions & 2.8224 $\pm$ 0.6810
        \end{tabular}
        \caption{Mean and standard deviation of ranks in each of the
        rubric dimensions we study.}
        \label{table:grade-stats}
    \end{centering}
\end{table}


\subsection{Ordinal Regression Feasibility Study}

Because of the ordinal nature of our grade labels (categorical with an
implicit ranking), it is natural to apply ordinal regression techniques to
our automated grading setup. These methods fall into the category of
supervised learning. In particular, we will utilize support vector
ordinal regression (SVOR)~\cite{Chu:2007:SVOR}, a generalization of the
popular support vector machine (SVM)~\cite{Cortes:1995:SVM} for
classification to the case of ordinal labels.

\subsubsection{Sensitivity to Training Size}
We first explored using only the most general of our feature types---token
features---to attempt to understand the sensitivity of the method to the
amount of student submissions labeled. Frequency-based token features were
extracted: we used the \textsc{MeTA}
toolkit\footnote{\url{https://meta-toolkit.org}} at version 1.1
with its default tokenizer, stemmer, and stopword list. For regression, we
used a modified version of
\textsc{LIBSVM}\footnote{\url{http://www.work.caltech.edu/~htlin/program/libsvm/}}
adapted to ordinal regression~\cite{Li:2007:NIPS}.

In an actual grading scenario, the instructor would manually grade a
certain number of the submissions, learn the regression function from these
labeled examples, and then apply the learned model to the remaining
unlabeled examples. To simulate this, we ran the following experiments: for
each rubric dimension, we took the collection of student documents and
randomly split it into two groups (the training and test sets) of specific
sizes, slowly increasing the size of the training set (and thus decreasing
the size of the test set). A function is learned based on the labeled
training set which is then used to label the examples in the test set. We
compute the \textbf{mean absolute error} (MAE), defined as
\[
MAE = \frac{1}{n} \sum_{i=1}^n | r(f(x_i)) - r(y_i) |
\]
where $f(x_i)$ is the predicted label of the example $x_i$, $r(\cdot)$ is
the rank of a given label, and $y_i$ is the gold standard label for the
example $x_i$. This experiment is repeated for ten different randomized
splits, and we report the average and standard deviation of the test set
MAE in Table~\ref{table:train-set-size}. Note that due to the difference in the
test set, the results from different sizes of training sets are not strictly
comparable.

\begin{table}
    \begin{centering}
        \begin{tabular}{l|c}
            analysis
            &
            \begin{tabular}{@{}c@{}}
%                0.6245 $\pm$ 0.1089\\
                0.5642 $\pm$ 0.0733
            \end{tabular}
            \\\hline

            answers
            &
            \begin{tabular}{@{}c@{}}
%                0.5623 $\pm$ 0.0290\\
                0.5491 $\pm$ 0.0634
            \end{tabular}
            \\\hline

            application
            &
            \begin{tabular}{@{}c@{}}
%                0.3170 $\pm$ 0.0375\\
                0.3321 $\pm$ 0.0580
            \end{tabular}
            \\\hline

            clarity
            &
            \begin{tabular}{@{}c@{}}
%                0.7321 $\pm$ 0.0777\\
                0.7604 $\pm$ 0.0659
            \end{tabular}
            \\\hline

            quality
            &
            \begin{tabular}{@{}c@{}}
%                0.9736 $\pm$ 0.143\\
                0.7868 $\pm$ 0.0810
            \end{tabular}
            \\\hline

            questions
            &
            \begin{tabular}{@{}c@{}}
%                0.5094 $\pm$ 0.1439\\
                0.4321 $\pm$ 0.0640
            \end{tabular}
        \end{tabular}
        \caption{Difficulty of grading each rubric dimension, characterized
        by MAE of a SVOR model learned on 50\% of the data. 10 randomized
        experiments were run; reported is the average and standard deviation.}
        \label{table:train-set-size}
    \end{centering}
\end{table}


There are a few things that we can observe from these results.  Perhaps
unsurprisingly, the rubric labels with the least variation are the easiest
to predict (e.g., ``application'' and ``questions'')---these dimensions
seem to benefit the least from the inclusion of more training examples.
The more difficult dimensions with higher data variance (e.g.,
``quality'') benefit more strongly from the inclusion of more training
data---more examples allow the method to better capture the
\emph{generalizable} patterns that apply to unseen examples.

\subsubsection{Overfitting}
A concern with complicated models such as SVMs, especially when applied to
small datasets with high dimensionality, is that of overfitting: the
training data is essentially memorized, causing the output function to
fail to perform well on new examples due to its failure to capture
patterns that generalize~\cite{Dietterich:1995:ACMCS}. Feature selection is
a method that can combat this in which only a subset of the possible
features in a given type are used~\cite{Guyon:2003:JMLR}. In our setup, we had
2646 total token features, which we attempted to reduce by only considering
tokens that had a total collection frequency above a certain threshold $k \in
\{10, 20, 100, 200\}$ resulting in 558, 383, 144, and 61 features,
respectively.  Unfortunately, this selection method failed to reduce the
overfitting phenomena, likely due to data sparsity. It would be worth exploring
more principled feature selection in future work, as this should in principle
improve the model's generalization.

\subsubsection{The Impact of Different Feature Types}
Moving beyond simple token features, we extracted both similarity and
selection features from our assignments and incorporated them
incrementally into our model to measure the predictive capacity of
different feature types.

Our token features were generated using the same process detailed
previously (frequency-based features extracted using the \textsc{MeTA}
toolkit).

Our similarity features (compared against an instructor-generated
assignment submission) were overall similarity, similarity of only
observation bullets, and similarity of only analysis bullets---these were
computed using the Okapi BM25 similarity function often used in information
retrieval as a scoring function~\cite{Robertson:1994:SIGIR,
Robertson:1996:TREC-3}. The instructor submission was treated as a query
and the student submissions as documents to be scored by utilizing the
following function:
\begin{align*}
    score(Q, D) &= \sum_{w \in Q \cup D} QTF(w, Q) \cdot TF(w, D) \cdot
IDF(w)\\
\intertext{where}
    QTF(w, Q) &= \frac{(k_3 + 1)c(w, Q)}{k_3 + c(w, Q)},\\
    TF(w, D) &= \frac{(k_1 + 1)c(w,D)}{k_1 + ((1-b) + b \cdot
\frac{|D|}{avdl}) + c(w,D)},\\
    IDF(w) &= \frac{N - df(w) + 0.5}{df(w) + 0.5},
\end{align*}

and $c(w, Q)$ is the frequency of the word $w$ in the query, $c(w, D)$ is
the frequency of the word $w$ in the document to be scored, $|D|$ is the
length of the document, $avdl$ is the average document length, $N$ is the
total number of documents, and $df(w)$ is the number of documents in which
the term $w$ appears. $k_1 = 1.2$, $k_3 = 500$, and $b = 0.75$ are
parameters which were set to heuristic defaults. This function has been
shown to be very robust, capturing a large number of important information
retrieval heuristics~\cite{Fang:2005:Axiomatic}, motivating our choice.

Our selection features were precision and recall of the
selected lab data in the student case analysis when compared against the
instructor's assignment.

We investigate the predictive capacity of these features by exploring the
improvement of the model when predicting our most challenging labels
(``quality'' and ``clarity''). We ran ten separate experiments with
different randomized training sets consisting of 50\% of the data when
using different feature combinations to represent the student submissions.
We again report the average and standard deviation of the MAE on the test
set across the ten runs. To further explore whether the regression method
is truly capturing patterns relevant for grading, we compare its MAE
against the MAE obtained by using a na\"ive baseline: compute the most
frequent label in the training data, and then assign this label to all
examples in the test set.  Intuitively, this is a very reasonable baseline
when comparing MAE---if the labels are normally distributed, picking the
most frequent one will ensure an absolute error of zero for the majority of
the examples---while simultaneously being unhelpful for discriminative
grading (which the regression method hopes to capture). Our results are
summarized in
Tables~\ref{table:feature-comb}~and~\ref{table:feature-comb-clar}.

\begin{table}
    \begin{tabular}{r|l|l}
        & \textbf{Baseline} & \textbf{SVOR}\\\hline

        \textbf{sim} (3)
        & 0.9358 $\pm$ 0.0882
        & \textbf{0.8811 $\pm$ 0.0940}
        \\\hline

        \textbf{sim + sel} (5)
        & 0.9566 $\pm$ 0.1677
        & \textbf{0.8642 $\pm$ 0.0325}
        \\\hline

        \textbf{toks} (2646)
        & 0.9075 $\pm$ 0.0789
        & \textbf{0.7660 $\pm$ 0.0910$^\dagger$}
        \\\hline

        \textbf{all} (2651)
        & 0.9792 $\pm$ 0.1568
        & \textbf{0.7566 $\pm$ 0.0738$^\dagger$}
        \\\hline
    \end{tabular}
    \\\\
    {\footnotesize
    $\dagger$: statistically signifigant using an unpaired $t$-test with $p
    \leq 0.05$.}

    \caption{Effectiveness (in terms of MAE) of incorporating additional
    features in grade prediction for ``quality'' dimensions using SVOR
    methods compared to the mode-assigning baseline. Number of features is
    given in parenthesis.}
    \label{table:feature-comb}
\end{table}

\begin{table}
    \begin{tabular}{r|l|l}
        & \textbf{Baseline} & \textbf{SVOR}\\\hline

        \textbf{sim} (3)
        & 0.7906 $\pm$ 0.0771
        & \textbf{0.7830 $\pm$ 0.0836}
        \\\hline

        \textbf{sim + sel} (5)
        & \textbf{0.7623 $\pm$ 0.0649}
        & 0.7811 $\pm$ 0.0561
        \\\hline

        \textbf{toks} (2646)
        & 0.7528 $\pm$ 0.0550
        & \textbf{0.7415 $\pm$ 0.0597}
        \\\hline

        \textbf{all} (2651)
        & \textbf{0.7189 $\pm$ 0.0617}
        & 0.7226 $\pm$ 0.0527
        \\\hline
    \end{tabular}

    \caption{Similar experiment to Table~\ref{table:feature-comb}, but for
    ``clarity'' dimension.}
    \label{table:feature-comb-clar}
\vskip-10pt
\end{table}


We see that for the ``quality'' dimension, the model is able to
successfully learn generalizable patterns in our features to predict the
label with errors that are statistically significantly less than the
baseline method. In general, the token features dominate the performance,
but it would seem as though the similarity and selection features have
lower variability in the MAE. Again, this result suggests that there are
likely gains to be had by utilizing a more sophisticated feature selection
method to remove some of the noise introduced by extraneous token features.

However, the ``clarity'' label shows us that the problem is far from being
solved in a general sense. Here, we see that our method consistently fails
to beat the baseline method, with the winning method being seemingly
random. This indicates that the features we have selected thus far are more
tailored toward discrimination along certain dimensions of the grading
rubric than others. More work must be placed into developing features that
truly capture the ``clarity'' dimension to allow the model to extract the
patterns the instructor observes when grading along this dimension.

What this demonstrates is that automatic grading of complex assignments is
currently feasible, but perhaps only in a limited fashion. Careful feature
generation is required, but in some cases a model can be learned to
effectively grade assignments. We suspect that significant gains in
grading performance can be obtained in other dimensions through careful
feature generation.

\subsection{An Active Learning to Rank Approach}
Supervised learning, however, is not a particularly good fit for grading
practices in reality for a number of reasons. First, as we observed in
Tables~\ref{table:feature-comb}~and~\ref{table:feature-comb-clar}, outright
prediction of an ordinal grade can be very challenging due to the highly
concentrated nature of the dataset labels (see
Table~\ref{table:grade-stats}). The vast majority of grade information
available for the grade prediction task is centered around the mean,
leaving very little information in the tails for a supervised learner to
extract patterns from. (In some cases, for example, there are as few as one
example for the highest and lowest ordinal grade values). The result is
noisy output that may be inappropriate for using directly. However, it is
worth noting that ordinal grade prediction is a hard problem, even for
humans: a previous study suggests disagreement rates around $44\%$ for
short answer grading~\cite{Mohler:2009:EACL}. We suspect that this only
becomes larger as assignments become more complex and difficult to grade,
which makes the task of outright label prediction much more difficult for
the machine as well.

Perhaps a more reasonable approach, then, is to produce a ranked list of
assignments from best to worst. Annotators are typically more consistent at
providing judgments of the form ``is $a$ better than $b$?''\ than ``on a
scale from 1--5, how good is $a$?''~\cite{Callison-Burch:2007:WMT}, so it
is reasonable to suspect that a machine learning model could achieve better
results when trained using such pairwise judgments. If a system can provide
a good ranking of assignments, an instructor simply needs to assign
``cutoff'' points in this ranking to determine grades. This simplifies the
learning problem from attempting to predict an ordinal label for a specific
assignment to assigning a ranking to a set of assignments. This is a well
studied area in information retrieval called ``learning to
rank''~\cite{Joachims:2002:KDD}, and there are a wide variety of methods
available that one can use to learn a ranking function for documents given
a set of available features.

Second, a serious drawback of the supervised learning approach is the lack
of early feedback in the instructor--machine collaboration. In the
supervised approach, a model is learned on some subset of the assignments
to predict the grades of the remaining assignments, but the process of
selecting a subset of assignments for grading is not guided. It is possible
that randomly selecting assignments to grade results in mostly redundant
information being given to the grading model. Instead, a better approach
could be to employ active learning to allow the machine learning model to
guide the instructor in providing the supervision to make the most
effective use of his/her effort.

Building on these two observations, we propose the following ``pairwise
active learning to rank'' model for automatic grading, which will employ
the following process:
\begin{enumerate}
  \item Ask the instructor for comparative judgments on $k_1$ pairs of
    assignments.
  \item Learn a model using a learning-to-rank approach on the available
    pairwise judgments.\label{al:learn-model-step}
  \item Apply the model to all remaining unjudged pairs.
  \item Select an unjudged pair to present to the instructor for judgment.
    \label{al:select-training-step}
  \item Go to step~\ref{al:learn-model-step}.
\end{enumerate}
Instantiations of this general approach will differ mainly in
steps~\ref{al:learn-model-step}~and~\ref{al:select-training-step}.

Before we explore the efficacy of such an approach, however, we must first
redefine some measure by which we can measure performance. Because the
system is no longer predicting a rating for each assignment, we cannot use
MAE as before.

\subsubsection{Evaluating Ranking-based Grading Systems}
Our goal is to produce a ranking of student assignments that is consistent
with instructor evaluation. One way of framing this problem is to compare
the ranking produced by the system to the ranking produced by the
instructor (which we'll call the ``reference ranking''). A system's ranking
can then be evaluated using some measure of correlation between the two
rankings. We note a preference for metrics that take into account the
\emph{entire} ranked list---this contrasts with most of the preferred
measures in information retrieval evaluation which typically place heavier
emphasis on the top-ranked elements. While this makes sense in a search
context, our goal is to produce an exhaustive ranking of the assignments,
so we focus on these types of measures.

Measures for rank correlation are plentiful. Perhaps the most commonly
used metrics are Kendall's $\tau$ or Spearman's $\rho$ (which have been
found to be highly correlated in practice~\cite{Shani:2011:Springer}; thus,
we present only one for illustration). Kendall's $\tau$ can be formulated
as
\[
    \tau = \frac{n_c - n_d}{\frac{1}{2} n (n-1)},
\]
where $n_c$ is the number of \emph{concordant pairs}, and $n_d$ is the number
of \emph{discordant pairs}, and $n$ is the number of items ranked. To
compute $n_c$ and $n_d$, one considers all pairs $(x_i, y_i)$ and $(x_j,
y_j)$ (that is, pairs of tuples) of assigned rankings in the system ranking
$X$ and the reference ranking $Y$ (the denominator is simply the number of
such pairs). A pair is \emph{concordant} if the ordering of the items $i$
and $j$ in $X$ and $Y$ is consistent---in other words, if $(x_i < x_j)
\land (y_i < y_j)$ or $(x_i > x_j) \land (y_i > y_j)$.  A pair is
\emph{discordant} if the ordering of items in the two lists is
inconsistent---in other words, if $(x_i < x_j) \land (y_i > y_j)$ or $(x_i
> x_j) \land (y_i < y_j)$. This is then a correlation measure, with values
bounded in $[-1, 1]$, with $1$ indicating a perfect correlation and $-1$
indicating inverse correlation.

One of the assumptions Kendall's $\tau$ makes is that there are no ties in
ranks. However, in a realistic grading scenario based on rubrics we expect
many ties. Fortunately, there is a variation of Kendall's $\tau$, denoted
as $\tau_b$, that accounts for ties in the rankings. This is formulated as
\[
    \tau_b = \frac{n_c - n_d}{\sqrt{(n_c + n_d + t_x)(n_c + n_d + t_y)}}
\]
where $t_x$ is the number of pairs that were tied on \emph{only} their
ranking from $X$, and $t_y$ is the number of pairs that were tied on
\emph{only} their ranking from $Y$.

This may, at first glance then, seem like a good measure to use, but it is
not without its problems. Despite taking into account ties in the rankings,
it may still penalize a system for re-ordering items that were tied in the
reference ranking---in other words, we may be penalized for not correctly
identifying elements who are tied in the reference ranking. Consider a
simple example: suppose the ranking proposed by a system is $X =
(1,2,3,4,5,6)$ but the reference ranking is $Y = (1,1,2,2,3,4)$.
Intuitively, the system made no real mistakes in that no pair where the
reference ranking asserted an is in the wrong order in $X$. However, we'll
see that $\tau_b \approx 0.9309$, indicating that the system did not achieve
perfect correlation.

To address this issue, Yao~\cite{Yao:1995:JASIS} proposed the normalized
distance-based performance measure (NDPM), which computes a distance
between two rankings that is insensitive to a system's reordering of tied
elements in the reference ranking. NDPM is computed as
\[
    NDPM = \frac{2n_d + t_x}{2(n_c + n_d + t_x)}.
\]
Note that this is a \emph{distance} measure, so a value of $0$ would
indicate a perfect ranking. Indeed, if we compute NDPM for the example
rankings above, we achieve this result. Thus, we feel that NDPM is perhaps
the most appropriate measure for evaluating automatic grading systems that
produce an ordering of assignments as their output.

\subsubsection{Efficiently Utilizing Human Judgments with Active Learning}
To study whether our proposed active learning approach better utilizes
human judgments during the grading process, we performed the following
experiment. We took our assignments and assigned each a ``composite
score'', computed as the average of their ordinal score for each of the six
rubric dimensions. Our task is then to learn a ranking that is consistent
with the ranking produced by these composite scores while simultaneously
\emph{minimizing instructor effort} in providing the necessary supervision.

We first transform the $n = 107$ assignments into $\frac{1}{2}n(n-1) =
5671$ assignment \emph{pairs} $(x_i, x_j)$ with corresponding labels
$y_{ij} \in \{+1, -1\}$ indicating whether $x_i$ should be ranked above or
below $x_j$ in the ranking. Ties were broken arbitrarily by assignment id.
The supervision given by the instructor is then to indicate a preference
for ranking $x_i$ relative to $x_j$.

Following the process laid out in the beginning of the section, we first
start with $k_1 = 10$ random pairs selected from the transformed data and
ask for labels from the instructor. We then learn the model, compute the
NDPM for the ranking produced by the model for all $n$ assignments, and
then ask for additional supervision by selecting the unlabeled assignment
pair whose distance from the decision boundary for the model is lowest
(this is a known, simple approach to uncertainty
sampling~\cite{Settles:2012}) and repeat the training/evaluation loop. Our
particular model choice was a linear SVM provided through the \textsc{MeTA}
toolkit.

We compare this active learning scenario with two different baselines. The
first is a simple na\"ive approach that ranks all student assignments by
their BM25 similarity with the instructor document as a query. Our other
baseline is the exact same process as above, but instead of selecting the
most uncertain pair in the unlabeled data we select one uniformly at
random. This will allow us to see whether the uncertainty sampling approach
is truly helping to guide the learning process to make more efficient
supervision choices or not.

\begin{figure}
  \begin{center}
  \begin{tikzpicture}
    \begin{axis}[xlabel={human-labeled pairs},ylabel={NDPM}, xmax=500,
      legend style={font=\small}]

      \addplot+[color=red, each nth point={10}, filter discard
          warning=false, mark=triangle, error bars/.cd, y dir=both, y explicit]
        table [x=training-size, y=AVG-NDPM, y error=std-deviation, col
            sep=comma] {data/random-learning.csv};

      \addplot+[color=blue, each nth point={10}, filter discard
          warning=false, mark=x, error bars/.cd, y dir=both, y explicit]
        table [x=training-size, y=AVG-NDPM, y error=std-deviation, col
          sep=comma] {data/active-learning.csv};

      \addplot[mark=none, color=green]
        coordinates {(0, 0.344637) (500, 0.344637)};

      \legend{Random learning, Active learning, BM25 similarity}
    \end{axis}
  \end{tikzpicture}
  \caption{A comparison between a no-learning solution, a randomized
  learning solution, and an active learning solution to the
  grading-as-ranking problem. Reported is the average NDPM (lower is
  better) over 5 runs, with error bars indicating one standard deviation.}
  \label{fig:active-learning}
  \end{center}
\end{figure}

Our results are summarized in Figure~\ref{fig:active-learning}. We can
immediately observe that the learning-based approaches significantly
outperform the na\"ive text-similarity ranking baseline, which should be
expected. More importantly is the difference between the active learning
method (blue line) and the random learning method (red line). We can see
that even at a small fraction of all of the assignment pairs, the active
learning approach is able to achieve better NDPM than simply learning at
random. This is consistent with our hypothesis that active learning as part
of an automatic grading system can make more effective use of an
instructor's time than a purely passive supervised approach.

\begin{figure}
  \begin{center}
  \begin{tikzpicture}
    \begin{axis}[xlabel={investigated assignments},ylabel={NDPM}, xmax=100,
      legend style={font=\small}]

      \addplot+[scatter, only marks, mark=x]
        table [x=num-distinct, y=NDPM, col sep=comma]
        {data/results-num-examined1.csv};

      \addplot+[scatter, only marks, mark=triangle]
        table [x=num-distinct, y=NDPM, col sep=comma]
        {data/results-num-examined2.csv};

      \addplot+[scatter, only marks, mark=diamond]
        table [x=num-distinct, y=NDPM, col sep=comma]
        {data/results-num-examined3.csv};

      \addplot+[scatter, only marks, mark=circle]
        table [x=num-distinct, y=NDPM, col sep=comma]
        {data/results-num-examined4.csv};

      \addplot+[scatter, only marks, mark=o]
        table [x=num-distinct, y=NDPM, col sep=comma]
        {data/results-num-examined5.csv};

      \addplot[mark=none, color=green]
        coordinates {(0, 0.344637) (100, 0.344637)};
    \end{axis}
  \end{tikzpicture}
  \caption{The number of unique assignments an instructor must see in order
  to achieve a specific NDPM. Pictured are five different simulations, each
  with a different seed set of $k_1 = 10$ pairs. The green line corresponds
  to the same BM25 similarity baseline as in
  Figure~\ref{fig:active-learning}.}
  \label{fig:num-examined}
  \end{center}
\end{figure}

%\begin{figure}
%  \begin{center}
%  \begin{tikzpicture}
%    \begin{axis}[xlabel={human-labeled assignments},ylabel={NDPM}, xmax=50,
%      legend style={font=\small}, cycle list name=color list]
%
%      \addplot+[color=red, mark=triangle,
%          error bars/.cd, y dir=both, y explicit]
%        table [x=num-graded, y=AVG-NDPM,
%               y error=std-deviation,
%               col sep=comma] {data/results-assign-random.csv};
%
%      \addplot+[color=blue, mark=x,
%          error bars/.cd, y dir=both, y explicit]
%        table [x=num-graded, y=AVG-NDPM,
%          y error=std-deviation,
%          col sep=comma] {data/results-assign-lc-single.csv};
%
%      \addplot+[color=orange, mark=square,
%          %error bars/.cd, y dir=both, y explicit]
%          ]
%        table [x=num-graded, y=NDPM,
%          %y error=std-deviation,
%          col sep=comma] {data/results-assign-lc-total1.csv};
%
%      \addplot+[color=brown, mark=diamond,
%          %error bars/.cd, y dir=both, y explicit]
%          ]
%        table [x=num-graded, y=NDPM,
%          %y error=std-deviation,
%          col sep=comma] {data/results-assign-lc-pair1.csv};
%
%      \addplot[mark=none, color=green]
%        coordinates {(0, 0.344637) (500, 0.344637)};
%
%      \legend{Random learning, Active learning, BM25 similarity}
%    \end{axis}
%  \end{tikzpicture}
%  \caption{A comparison between a no-learning solution, a randomized
%  learning solution, and an active learning solution to the
%  grading-as-ranking problem, using assignment grades as feedback instead
%  of pairwise judgments.}
%  \label{fig:assignment-based}
%  \end{center}
%\end{figure}


\section{Discussion and Future Work}

In the previous experiments we formulated the automated grading problem as
a ranking problem, and introduced a rank distance measure (NDPM) as a form
of evaluating the quality of a ranked list generated by an automated (or
semi-automated, in our case) system. Under a ranking-based problem
formulation, we argue that this is the most sensible metric for evaluating
performance relative to a gold standard.

However, the value of NDPM cannot be easily compared with the values of
existing metrics (such as MAE) that have been traditionally used in
evaluating automated grading systems in the past. There is a need to
evaluate a ranked list from the perspective of its impact on the eventual
grades assigned to student work. Unfortunately, how to evaluate the utility
of a ranked list appropriately remains a challenge partly due to the
difficulty in choosing the cutoffs, which may depend on the desired
tradeoff that an instructor wants (e.g., a desired distribution of grades
in different brackets). In practice, we envision that the instructor would
visit points in the ranked list and choose cutoffs based on the tradeoff
between the different types of grading errors. Exploring grade cutoff
assignment strategies remains an important future direction, and our
framework coupled with such a cutoff strategy would enable evaluation based
on the traditional grade prediction task.

While we believe the results here show that the methods employed are
feasible for grading complex assignments, more work remains to be done to
understand just how well our system performs relative to human judgments.
Future work should explore this by measuring human consensus in grading
these complex assignments, similar to what has done for short
answers~\cite{Mohler:2009:EACL}. Furthermore, we only investigated very
simplistic features---such as the bag-of-words model---which are very
general but not very sophisticated. Exploring the feature space further to
find more sophisticated features that perform well in practice and are more
tailored to the goals of medical case assessments remains as future work.

Another major limitation of our study is the limited size of the data set.
This is partly due to the fact that such complex assignments currently can
only be graded by human graders. In the future, we hope to deploy our
automated grading tools to help scale up such courses to enable more
students to participate, which in turn, would help collecting more data for
experiments.

Finally, a crucial direction that remains unexplored is feedback: how could
such a system give more detailed feedback to students beyond just their
ordinal rating along a rubric dimension? Currently, peer grading approaches
have an advantage in this sense, as your peers can suggest to you
corrections or point out specific mistakes that you made. It is worth
investigating whether or not we can generate ``explanatory reports'' of
grading results when using a supervised learning approach.

\section{Conclusions}

Automated grading of complex assignments is necessary for scaling up
learning without compromising effectiveness of learning. Using a data set
of medical case assessment assignments, we conducted the first systematic
study of how we might be able to use machine learning to automate grading
of such a complex assignment. Our study has led to several contributions.

First, we have experimentally shown the feasibility of using supervised
learning techniques for automated grading of medical case assignments under
certain conditions provided that the instructor can manually label a number
of the assignments to serve as a training set. In particular, an ordinal
regression method can be applied to the data with results that consistently
outperform the majority-label baseline in terms of mean absolute error.

Second, we proposed a general framework for the development of three
complementary types of representative features for student submissions
(i.e., token features, similarity features, and selection features)
---while we applied these features to our specific task of medical case
analysis grading, these feature types (and generation framework) are
general and should apply to the grading of any complex assignment.

Third, we proposed to frame the problem of automated grading as a ranking
problem, which can more naturally assist human graders to validate and
finalize grades of ungraded assignments and learn from pairwise preference
judgments that can be potentially created more reliably by human graders
including through peer grading. We also suggested NDPM as potentially a
better measure for this ranking task than other measures due to its
superiority in handling many tied cases.

Finally, we proposed an iterative procedure of online active learning to
rank to efficiently utilize human judgments, and thus optimizing the
collaboration between human graders and the automated grader. Experiment
results confirm the efficiency of this procedure which can substantially
save human effort as compared with randomly choosing sample assignments for
humans to grade.





\section{Acknowledgments}

Anonymized for blind reviewing.

\balance

\bibliographystyle{acm-sigchi}
\bibliography{auto-grading}
\end{document}
