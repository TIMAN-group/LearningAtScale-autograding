\section{Medical Case Assignment}

Complex assignments inevitably vary across courses. As a first step toward
studying how to automate grading of such assignments, we  use a medical
case assignment in the veterinary medicine domain for our study. At a high
level, such an assignment represents a typical type of analysis assignment
where the students are given a case description with both an unstructured
text description as well as some structured data (e.g., lab test results),
and are asked to perform an analysis of the case. The analysis generally
involves 1) selecting relevant content from the case description, which can
be selected from both the text part and the structured data, 2) answering
questions with short textual answers, and 3) writing assessments in
natural language text.

More specifically, the case exercises were developed using
WhenKnowingMatters (WKM) web-based case formulation
software{\footnote{available at \url{http://www.whenknowingmatters.com}.}
which facilitates development and exchange of text-based cases while
allowing students to objectify their observations from a case and
manipulate them in an outline format around a suggested scaffold provided
by the instructor. The student's analysis is then rendered into a
structured text format to facilitate automatic grading.

\begin{figure*}[ht]
\includegraphics[scale=0.5]{case-desc-small.png}
\includegraphics[scale=0.6]{student-work-small.png}
\caption{An example of case description (left) and student assessment (right)}
\label{fig:example}
\end{figure*}

Due to the lack of automated grading tools, the assignments are currently
graded manually. An assessment rubric designed prior to instruction was
used by the instructor to evaluate student performance on a subjective,
5-point scale (listed here in increasing order): novice, beginner,
competent, proficient, and expert. Rubric categories were related to
elements of critical thinking and communication:
\begin{enumerate}
\item {\bf Questions:} Developing relevant refining (or clarifying)
 questions to answer based upon an honest assessment of current knowledge
 base.
\item {\bf Answers:} Approach to seeking answers to developed
 questions; literature search, etc.
\item {\bf Quality:} Judgment of quality of information; awareness and
 application of standards of a discipline, bias detection including
 appropriate humility to detect one’s own potential bias, application of
 statistical concepts.
\item {\bf Analysis:} Analysis of an argument.
\item  {\bf Clarity:} Clarity and communication of thought; conciseness,
 grammar, spelling, elocution.
\item {\bf Application:} Application and understanding of appropriate
 disciplinary content.
\end{enumerate}

The instructor also created a ``gold standard" assessment for the
assignment case, which is available for the automated grading tool to use.

Figure~\ref{fig:example} shows a sample case description and a typical
student answer. In the case description, the student can see a text
description of the case and a number of lab test results in the form of
structured data. The student assessment is seen to be a semi-structured
text with indented structures based on a scaffold provided by the
instructor. Multiple tags indicate different kinds of answers, including,
e.g., selected content from the original case description, selected lab
tests, and text input by the student reflecting his/her assessment.

Because of the complexity, automated grading of such an assignment is very
challenging. First, because of variations across different assignments, it
is almost impossible to learn from the grading results of one assignment to
automate grading of another, even though such an ``inter-assignment"
automated grading is ideal.  We thus focus on a more realistic setting of
attempting to automate the grading after the instructor has already graded
some assignments, which we may refer to as ``intra-assignment" automated
grading, which, strictly speaking, is actually semi-automatic grading.  Our
goal is thus to study whether and how we can leverage machine learning to
learn from the graded assignments to automatically predict the grades for
the rest of the ungraded assignments.
