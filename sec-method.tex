\section{Feasibility of Automated Gradiing}

In this section, we discuss and study the feasibility of using machine learning methods, particularly supervised learning,  for automating the grading of complex assignments. We first present the general idea of supervised learning, then propose a general methodology for designing three complementary types of features for representing assignments, which are needed for supervised learning, and finally present experiment results.

\ignore{
\subsection{Unsupervised Learning}
%
Our first thought is to use unsupervised learning since it does not require any training data (i.e., graded assignments as examples).  In unsupervised learning, a model is built to describe some latent
structure of the data without considering any label information. Typical
examples of unsupervised methods include clustering and topic modeling.
Unsupervised methods have been applied to automated grading typically in
the form of clustering---the ``latent structure'' to be extracted are
groups of student submissions, which can then be collectively assigned a
single grade. Hierarchical clustering methods allow an instructor to
``drill down'' by dividing a single cluster into several smaller ones that
somehow capture differences within the same group. This method was
exploited to great effect by the power grading
project~\cite{Brooks:2014:Powergrading} to allow instructors to spend
significantly less time grading short answer questions.
%
However, this does not address the need to evaluate students along
different dimensions as occurs in rubric grading for complex assignments.  Because the method does
not consider the labels in learning the latent structure, the structure it
finds will tend to be general with no principled way of tuning it to
better describe the desired label outcomes. Furthermore, this strategy
also provides no guidance for the instructor in assigning the actual grade
value itself---while it aids in digesting the patterns that occur in the
data, the instructor is still on his/her own in deciphering what the grade
should be based on these patterns. With simple questions with more or less
one correct answer, this is not a problem---but when we attempt to address
more complex assignments that involve critical thinking and principled
analysis (for which there are many ``correct'' answers) this becomes much
more difficult and time consuming.
}

\subsection{Supervised Learning}

In supervised learning, a model is built to predict the outcome (or label)
of a new data example based on previous examples it has seen before (called the
training data). Thus a natural way to use supervised learning for grading
is to have a human (e.g., instructor) to grade a set of assignments to be used
as training data to learn a model to predict the grade of each ungraded
assignment. A critical component of this infrastructure is the
decomposition of examples into feature vectors---this decomposition enables
the use of algorithms for learning functions from these vectors to the
output labels desired. Typically, these feature vectors are either binary
or real-valued, and are often (but not always) in a high-dimensional space.
The performance of the learned function is directly tied to the features
used in the vector representation of the examples---poor features result
in low predictive capability due to the algorithm being unable to find
meaningful patterns in the examples. As such, these features are a
critical component of any supervised learning approach. With a properly
defined set of features that are capable of capturing the salient patterns
in the training examples, the task can be given to any of a number of
state-of-the-art algorithms for learning appropriate predictive functions
that can be applied to yet-unseen data (the test data). Another factor
affecting the accuracy of prediction is the number of training examples, with
more training examples leading to higher accuracy. However, since
creating training examples generally requires manual work, we tend
to have only a limited number of training examples to work with.
How to define general features that we can automatically compute
based on a complex assignment and how to minimize manual effort
in creating training examples are two major questions that we study in this paper.

\ignore{This scenario is applicable to an automated grading setup in which the
instructor labels a certain number of student assignments with a grade
which are then fed in to the algorithm to learn a predictive function that,
when applied to the unlabeled examples, provides a grade based on the
patterns extracted from the manually graded examples.
%
This supervised framework would allow instructors to learn a separate
predictor for each dimension in a rubric by simply changing the labels of
the training examples to reflect a different rubric dimension during
training. We will explore supervised learning approaches in this paper to
address our need for predicting different rubric dimensions instead of a
single overall grade. In such a setting, the grading problem has been
reduced to defining a set of features that characterize a student's
assignment along multiple dimensions.}

\subsubsection{Defining Features of a Student Assignment}

The performance of a supervised learning approach is highly dependent on
the effectiveness of the features fed into the learning program. Thus a main technical
challenge we need to solve is how to design effective features for representing
an assignment.

To address this challenge, we propose a general framework for defining features for complex
assignments such as the one we explore in this paper. The features we
propose are general in nature and thus should be applicable to any
assignment that is presented in a text-based, semi-structured response
form. We describe a set of feature classes and evaluate the performance of
these features on an example autograding task to evaluate their predictive
capacity. Our framework consists first of constructing a ``view'' of an
assignment and then defining features based on this view. The view chosen
for the assignment is critical in that it changes the way we may naturally
describe it and thus leads to the definition of distinct classes of
features distinguished by the view taken to derive them. We will explore
features by progressively taking views that make stronger assignment design
assumptions: while the features are still general, each view progressively
narrows the space of possible student response types.

The first class of features, which we call \textbf{token features}, are
generated by taking a view of the student response consistent with the
traditional ``bag of words'' approach common in information retrieval
contexts~\cite{Manning:2008}. In this view a document is decomposed into
a vector of count data that indicates the frequency of words within the
document. Two features are thus natural. The first type of feature would
indicate the number of occurrences of a given word in a student submission
(and is thus real-valued), and the second would indicate the presence or
absence of a word (and is thus binary-valued). These features would both
create a high-dimensional representation of the student submission, and are
motivated by an attempt to capture the difference in vocabulary between
assignments. This is often enough to capture whether the correct ideas are
mentioned without requiring extensive computation (features from this class
are trivial to compute for every student submission), though more discriminative
units such as n-grams (a sequence of $n$ words) may also be easily used
to replace single words if necessary.  Document
classification techniques typically operate in this kind of space.

The second class of features, which we call \textbf{similarity features},
are generated by characterizing a student submission by the ``distance''
from a gold standard (e.g., an assignment submission generated by the
instructor). With this view, features can be derived that strongly utilize
the structure of the assignment (e.g., how closely does the outline
structure of the student assignment match the outline structure of the
instructor assignment?)\ as well as features that loosely utilize or
completely ignore the structure of the assignment. Examples of features
that loosely utilize the assignment structure would be the similarity of
certain outline bullet types with the gold standard bullet types of the
same category. A bullet type in our examples could be ``observation''
(indicating something selected from the assignment text directly) or
``analysis'' (indicating original thoughts from the student). These
features require the assignment to be structured in such a way that this
information is easily extracted, but do not look so closely at the overall
structure of the outline itself. Ignoring the structure of the assignment,
features can be generated that indicate the overall similarity with the
gold standard. Document clustering techniques typically operate in this
kind of space, as well as retrieval functions in search
systems~\cite{Manning:2008}.

The third class of features, which we call \textbf{selection features},
are generated by measuring concrete statistics about the selection of
bullet points compared to a gold standard. In some sense, these are similar
to the similarity features, but they differ in that they make a stronger
assumption about the assignment structure---namely, that students are
producing the exact same text that should occur in a similar section of the
gold standard. Examples of selection features would be precision (what
fraction of the bullets selected by the student also appear in the gold
standard?)\ and recall (what fraction of bullets selected in the gold
standard were also selected by the student?)~\cite{Manning:2008}.

\ignore{
\subsection{Active Learning}
One critical problem in the supervised learning setting is the selection
of a training set. If a non-representative training set is selected, the
algorithm has no opportunity to learn the features that distinguish the
excellent submissions from the ordinary submissions. Unfortunately, the
traditional supervised learning setting offers no principled mechanism for
picking which training set to use---it just assumes one exists a priori.
%
Active learning methods~\cite{Settles:2012} bridge this gap by providing a
mechanism for selecting relevant training examples designed to maximally
improve the performance of an existing model. This setting is very relevant
for an autograding setup, where the system should ideally ask the
instructor to grade a \emph{specific} set of examples, rather than forcing
the instructor to find good representative examples on his/her own. This
process can be iterative: the system can learn from the first batch of
examples graded by the instructor, and then request him/her to grade a
second batch, which is used to incrementally improve the learned model.
This should, in principle, reduce the amount of time an instructor would
have to spend grading to obtain a certain performance threshold for the
grade predictor.
%
\subsection{Combined Methods for Complete Grading Support}
%
A comprehensive system can then be designed that leverages all three of
these perspectives: unsupervised, (semi-) supervised, and active learning.
We can first apply an unsupervised learning technique to group the student
assignment data into rough categories. Then, a set of student submissions
can be sampled from these distinct categories, resulting in a collection of
submissions that are in some sense different from one another---these
examples are then labeled by the instructor and used as the starting point
for an active learning method: we can learn a supervised classifier using
the now labeled data, use this classifier to separate the remaining
unlabeled training data, sample more documents for the instructor to label,
and continue in an iterative fashion until either a fixed evaluation
criteria is met, or a certain number of submissions have been labeled.
Such a combined method would attempt to {\bf optimize the collaboration of
human graders and the automated grading tool}  so that we can leverage the
best of each.
}

\subsubsection{Ordinal Regression for Grade Prediction}

Because of the ordinal nature of our grade labels (categorical with an
implicit ranking), it is natural to apply ordinal regression techniques to
our automated grading setup. In particular, we will utilize support vector
ordinal regression (SVOR)~\cite{Chu:2007:SVOR}, a generalization of the
popular support vector machine (SVM)~\cite{Cortes:1995:SVM} for
classification to the case of ordinal labels in the study of feasibility of
grade prediction.

\subsection{Experiment Results}

We now present the results of ordinal regression on our medical assignment
data set to assess the effectiveness of the proposed features and examine
how effective such a state of the art learning method is for solving the
grading problem.

We first explored using only the most general of our feature types---token
features---to attempt to understand the differences in grading difficulty
across our different rubric dimensions. Frequency-based token features were
extracted: we used the \textsc{MeTA}
toolkit\footnote{\url{https://meta-toolkit.org}} at version 1.1
with its default tokenizer, stemmer, and stopword list. For regression, we
used a modified version of
\textsc{LIBSVM}\footnote{\url{http://www.work.caltech.edu/~htlin/program/libsvm/}}
adapted to ordinal regression~\cite{Li:2007:NIPS}.

In an actual grading scenario, the instructor would manually grade a
certain number of the submissions, learn the regression function from these
labeled examples, and then apply the learned model to the remaining
unlabeled examples. To simulate this, we ran the following experiments: for
each rubric dimension, we took the collection of student documents and
randomly split it into two groups (the training and test sets) each
containing 50\% of the data\footnote{We do not use something like 10-fold
cross validation due to the small size of the available labeled data to
ensure that the training and test sets can be as representative of the
actual data as possible.}. A function is learned based on the labeled
training set which is then used to label the examples in the test set. We
compute the \textbf{mean absolute error} (MAE), defined as
\[
MAE = \frac{1}{n} \sum_{i=1}^n | r(f(x_i)) - r(y_i) |
\]
where $f(x_i)$ is the predicted label of the example $x_i$, $r(\cdot)$ is
the rank of a given label, and $y_i$ is the gold standard label for the
example $x_i$. This experiment is repeated for ten different randomized
splits, and we report the average and standard deviation of the test set
MAE in Table~\ref{table:train-set-size}.

\begin{table}
    \begin{centering}
        \begin{tabular}{l|c}
            analysis
            &
            \begin{tabular}{@{}c@{}}
%                0.6245 $\pm$ 0.1089\\
                0.5642 $\pm$ 0.0733
            \end{tabular}
            \\\hline

            answers
            &
            \begin{tabular}{@{}c@{}}
%                0.5623 $\pm$ 0.0290\\
                0.5491 $\pm$ 0.0634
            \end{tabular}
            \\\hline

            application
            &
            \begin{tabular}{@{}c@{}}
%                0.3170 $\pm$ 0.0375\\
                0.3321 $\pm$ 0.0580
            \end{tabular}
            \\\hline

            clarity
            &
            \begin{tabular}{@{}c@{}}
%                0.7321 $\pm$ 0.0777\\
                0.7604 $\pm$ 0.0659
            \end{tabular}
            \\\hline

            quality
            &
            \begin{tabular}{@{}c@{}}
%                0.9736 $\pm$ 0.143\\
                0.7868 $\pm$ 0.0810
            \end{tabular}
            \\\hline

            questions
            &
            \begin{tabular}{@{}c@{}}
%                0.5094 $\pm$ 0.1439\\
                0.4321 $\pm$ 0.0640
            \end{tabular}
        \end{tabular}
        \caption{Difficulty of grading each rubric dimension, characterized
        by MAE of a SVOR model learned on 50\% of the data. 10 randomized
        experiments were run; reported is the average and standard deviation.}
        \label{table:train-set-size}
    \end{centering}
\end{table}


We can observe that the rubric labels with the least variation are the
easiest to predict (e.g., ``application'' and ``questions''), whereas
rubric dimensions with higher data variance (e.g., ``quality'') are more
difficult.

\ignore{
\subsubsection{Overfitting}
A concern with complicated models such as SVMs, especially when applied to
small datasets with high dimensionality, is that of overfitting: the
training data is essentially memorized, causing the output function to
fail to perform well on new examples due to its failure to capture
patterns that generalize~\cite{Dietterich:1995:ACMCS}. Feature selection is
a method that can combat this in which only a subset of the possible
features in a given type are used~\cite{Guyon:2003:JMLR}. In our setup, we had
2646 total token features, which we attempted to reduce by only considering
tokens that had a total collection frequency above a certain threshold $k \in
\{10, 20, 100, 200\}$ resulting in 558, 383, 144, and 61 features,
respectively.  Unfortunately, this selection method failed to reduce the
overfitting phenomena, likely due to data sparsity. It would be worth exploring
more principled feature selection in future work, as this should in principle
improve the model's generalization.}

\subsubsection{The Impact of Different Feature Types}
Moving beyond simple token features, we extracted both similarity and
selection features from our assignments and incorporated them
incrementally into our model to measure the predictive capacity of
different feature types.

Our token features were generated using the same process detailed
previously (frequency-based features extracted using the \textsc{MeTA}
toolkit). Our similarity features (compared against an instructor-generated
assignment submission) were overall similarity, similarity of only
``observation'' bullets, and similarity of only ``analysis'' bullets.
These were computed using the Okapi BM25 similarity function often used in
information retrieval as a scoring function~\cite{Manning:2008}, treating
the instructor submission as a query and the student submissions as
documents to be scored. Finally, our selection features were precision and
recall~\cite{Manning:2008} of the selected lab data in the student case
analysis when compared against the instructor's assignment.

We investigate the predictive capacity of these features by exploring the
improvement of the model when predicting our most challenging labels
(``quality'' and ``clarity''). We ran ten separate experiments with
different randomized training sets consisting of 50\% of the data when
using different feature combinations to represent the student submissions.
We again report the average and standard deviation of the MAE on the test
set across the ten runs. To further explore whether the regression method
is truly capturing patterns relevant for grading, we compare its MAE
against the MAE obtained by using a na\"ive baseline: compute the most
frequent label in the training data, and then assign this label to all
examples in the test set.  Intuitively, this is a very reasonable baseline
when comparing MAE---if the labels are normally distributed, picking the
most frequent one will ensure an absolute error of zero for the majority of
the examples---while simultaneously being unhelpful for discriminative
grading (which the regression method hopes to capture). Our results are
summarized in
Tables~\ref{table:feature-comb}~and~\ref{table:feature-comb-clar}.


We see that for the ``quality'' dimension, the model is able to
successfully learn generalizable patterns in our features to predict the
label with errors that are statistically significantly less than the
baseline method. In general, the token features dominate the performance,
but it would seem as though the similarity and selection features have
lower variability in the MAE. Again, this result suggests that there are
likely gains to be had by utilizing a more sophisticated feature selection
method to remove some of the noise introduced by extraneous token features.

However, the ``clarity'' label shows us that the problem is far from being
solved in a general sense. Here, we see that our method consistently fails
to beat the baseline method, with the winning method being seemingly
random. This indicates that the features we have selected thus far are more
tailored toward discrimination along certain dimensions of the grading
rubric than others. More work must be placed into developing features that
truly capture the ``clarity'' dimension to allow the model to extract the
patterns the instructor observes when grading along this dimension.

\begin{table}
    \begin{tabular}{r|l|l}
        & \textbf{Baseline} & \textbf{SVOR}\\\hline

        \textbf{sim} (3)
        & 0.9358 $\pm$ 0.0882
        & \textbf{0.8811 $\pm$ 0.0940}
        \\\hline

        \textbf{sim + sel} (5)
        & 0.9566 $\pm$ 0.1677
        & \textbf{0.8642 $\pm$ 0.0325}
        \\\hline

        \textbf{toks} (2646)
        & 0.9075 $\pm$ 0.0789
        & \textbf{0.7660 $\pm$ 0.0910$^\dagger$}
        \\\hline

        \textbf{all} (2651)
        & 0.9792 $\pm$ 0.1568
        & \textbf{0.7566 $\pm$ 0.0738$^\dagger$}
        \\\hline
    \end{tabular}
    \\\\
    {\footnotesize
    $\dagger$: statistically signifigant using an unpaired $t$-test with $p
    \leq 0.05$.}

    \caption{Effectiveness (in terms of MAE) of incorporating additional
    features in grade prediction for ``quality'' dimensions using SVOR
    methods compared to the mode-assigning baseline. Number of features is
    given in parenthesis.}
    \label{table:feature-comb}
\end{table}

\begin{table}
    \begin{tabular}{r|l|l}
        & \textbf{Baseline} & \textbf{SVOR}\\\hline

        \textbf{sim} (3)
        & 0.7906 $\pm$ 0.0771
        & \textbf{0.7830 $\pm$ 0.0836}
        \\\hline

        \textbf{sim + sel} (5)
        & \textbf{0.7623 $\pm$ 0.0649}
        & 0.7811 $\pm$ 0.0561
        \\\hline

        \textbf{toks} (2646)
        & 0.7528 $\pm$ 0.0550
        & \textbf{0.7415 $\pm$ 0.0597}
        \\\hline

        \textbf{all} (2651)
        & \textbf{0.7189 $\pm$ 0.0617}
        & 0.7226 $\pm$ 0.0527
        \\\hline
    \end{tabular}

    \caption{Similar experiment to Table~\ref{table:feature-comb}, but for
    ``clarity'' dimension.}
    \label{table:feature-comb-clar}
\vskip-10pt
\end{table}


What this demonstrates is that automatic grading of complex assignments is
currently feasible, but perhaps only in a limited fashion. Careful feature
generation is required, but in some cases a model can be learned to
effectively grade assignments. We suspect that significant gains in
grading performance can be obtained in other dimensions through careful
feature generation.
