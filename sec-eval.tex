\section{Experiments}

In this section, we perform some experiments that focus specifically on the
(semi-) supervised methods detailed in the previous section, which we view
as the starting point towards a more comprehensive method for grading
complex assignments. We detail our empirical findings on an example
dataset to evaluate our proposed methods.

\subsection{Dataset}

Our data consist of $n = 107$ student submissions for one medical case
analysis assignment in a veterinary medicine course at XXXX institution.
Each was graded along six different rubric dimensions (abbreviations
bolded): \textbf{analysis} of argument, approach in seeking
\textbf{answers}, \textbf{application} of disciplinary content,
\textbf{clarity} of communication, judgment of \textbf{quality} of
information, and developing relevant refining \textbf{questions}. Student
submissions were given an ordinal rating on a five level scale from novice
(1) to expert (5) along each dimension---we report the mean rank and
standard deviation for each of the six labels in
Table~\ref{table:grade-stats}. The ordinal regression task is then to
learn a separate function to predict the rating for each of the six
dimensions using features extracted from the student documents.

\begin{table}
    \begin{centering}
        \begin{tabular}{l|c}
            analysis & 2.6355 $\pm$ 0.7660
            \\\hline
            answers & 3.0280 $\pm$ 0.7668
            \\\hline
            application & 2.8692 $\pm$ 0.5651
            \\\hline
            clarity & 3.3832 $\pm$ 0.9437
            \\\hline
            quality & 3.1121 $\pm$ 0.9795
            \\\hline
            questions & 2.8224 $\pm$ 0.6810
        \end{tabular}
        \caption{Mean and standard deviation of ranks in each of the
        rubric dimensions we study.}
        \label{table:grade-stats}
    \end{centering}
\end{table}


\subsection{Ordinal Regression Feasibility Study}

Because of the ordinal nature of our grade labels (categorical with an
implicit ranking), it is natural to apply ordinal regression techniques to
our automated grading setup. These methods fall into the category of
supervised learning. In particular, we will utilize support vector
ordinal regression (SVOR)~\cite{Chu:2007:SVOR}, a generalization of the
popular support vector machine (SVM)~\cite{Cortes:1995:SVM} for
classification to the case of ordinal labels.

\subsubsection{Sensitivity to Training Size}
We first explored using only the most general of our feature types---token
features---to attempt to understand the sensitivity of the method to the
amount of student submissions labeled. Frequency-based token features were
extracted: we used the \textsc{MeTA}
toolkit\footnote{\url{https://meta-toolkit.org}} at version 1.1
with its default tokenizer, stemmer, and stopword list. For regression, we
used a modified version of
\textsc{LIBSVM}\footnote{\url{http://www.work.caltech.edu/~htlin/program/libsvm/}}
adapted to ordinal regression~\cite{Li:2007:NIPS}.

In an actual grading scenario, the instructor would manually grade a
certain number of the submissions, learn the regression function from these
labeled examples, and then apply the learned model to the remaining
unlabeled examples. To simulate this, we ran the following experiments: for
each rubric dimension, we took the collection of student documents and
randomly split it into two groups (the training and test sets) of specific
sizes, slowly increasing the size of the training set (and thus decreasing
the size of the test set). A function is learned based on the labeled
training set which is then used to label the examples in the test set. We
compute the \textbf{mean absolute error} (MAE), defined as
\[
MAE = \frac{1}{n} \sum_{i=1}^n | r(f(x_i)) - r(y_i) |
\]
where $f(x_i)$ is the predicted label of the example $x_i$, $r(\cdot)$ is
the rank of a given label, and $y_i$ is the gold standard label for the
example $x_i$. This experiment is repeated for ten different randomized
splits, and we report the average and standard deviation of the test set
MAE in Table~\ref{table:train-set-size}. Note that due to the difference in the
test set, the results from different sizes of training sets are not strictly
comparable.

\begin{table}
    \begin{centering}
        \begin{tabular}{l|c}
            analysis
            &
            \begin{tabular}{@{}c@{}}
%                0.6245 $\pm$ 0.1089\\
                0.5642 $\pm$ 0.0733
            \end{tabular}
            \\\hline

            answers
            &
            \begin{tabular}{@{}c@{}}
%                0.5623 $\pm$ 0.0290\\
                0.5491 $\pm$ 0.0634
            \end{tabular}
            \\\hline

            application
            &
            \begin{tabular}{@{}c@{}}
%                0.3170 $\pm$ 0.0375\\
                0.3321 $\pm$ 0.0580
            \end{tabular}
            \\\hline

            clarity
            &
            \begin{tabular}{@{}c@{}}
%                0.7321 $\pm$ 0.0777\\
                0.7604 $\pm$ 0.0659
            \end{tabular}
            \\\hline

            quality
            &
            \begin{tabular}{@{}c@{}}
%                0.9736 $\pm$ 0.143\\
                0.7868 $\pm$ 0.0810
            \end{tabular}
            \\\hline

            questions
            &
            \begin{tabular}{@{}c@{}}
%                0.5094 $\pm$ 0.1439\\
                0.4321 $\pm$ 0.0640
            \end{tabular}
        \end{tabular}
        \caption{Difficulty of grading each rubric dimension, characterized
        by MAE of a SVOR model learned on 50\% of the data. 10 randomized
        experiments were run; reported is the average and standard deviation.}
        \label{table:train-set-size}
    \end{centering}
\end{table}


There are a few things that we can observe from these results.  Perhaps
unsurprisingly, the rubric labels with the least variation are the easiest
to predict (e.g., ``application'' and ``questions'')---these dimensions
seem to benefit the least from the inclusion of more training examples.
The more difficult dimensions with higher data variance (e.g.,
``quality'') benefit more strongly from the inclusion of more training
data---more examples allow the method to better capture the
\emph{generalizable} patterns that apply to unseen examples.

\subsubsection{Overfitting}
A concern with complicated models such as SVMs, especially when applied to
small datasets with high dimensionality, is that of overfitting: the
training data is essentially memorized, causing the output function to
fail to perform well on new examples due to its failure to capture
patterns that generalize~\cite{Dietterich:1995:ACMCS}. Feature selection is
a method that can combat this in which only a subset of the possible
features in a given type are used~\cite{Guyon:2003:JMLR}. In our setup, we had
2646 total token features, which we attempted to reduce by only considering
tokens that had a total collection frequency above a certain threshold $k \in
\{10, 20, 100, 200\}$ resulting in 558, 383, 144, and 61 features,
respectively.  Unfortunately, this selection method failed to reduce the
overfitting phenomena, likely due to data sparsity. It would be worth exploring
more principled feature selection in future work, as this should in principle
improve the model's generalization.

\subsubsection{The Impact of Different Feature Types}
Moving beyond simple token features, we extracted both similarity and
selection features from our assignments and incorporated them
incrementally into our model to measure the predictive capacity of
different feature types.

Our token features were generated using the same process detailed
previously (frequency-based features extracted using the \textsc{MeTA}
toolkit).

Our similarity features (compared against an instructor-generated
assignment submission) were overall similarity, similarity of only
observation bullets, and similarity of only analysis bullets---these were
computed using the Okapi BM25 similarity function often used in information
retrieval as a scoring function~\cite{Robertson:1994:SIGIR,
Robertson:1996:TREC-3}. The instructor submission was treated as a query
and the student submissions as documents to be scored by utilizing the
following function:
\begin{align*}
    score(Q, D) &= \sum_{w \in Q \cup D} QTF(w, Q) \cdot TF(w, D) \cdot
IDF(w)\\
\intertext{where}
    QTF(w, Q) &= \frac{(k_3 + 1)c(w, Q)}{k_3 + c(w, Q)},\\
    TF(w, D) &= \frac{(k_1 + 1)c(w,D)}{k_1 + ((1-b) + b \cdot
\frac{|D|}{avdl}) + c(w,D)},\\
    IDF(w) &= \frac{N - df(w) + 0.5}{df(w) + 0.5},
\end{align*}

and $c(w, Q)$ is the frequency of the word $w$ in the query, $c(w, D)$ is
the frequency of the word $w$ in the document to be scored, $|D|$ is the
length of the document, $avdl$ is the average document length, $N$ is the
total number of documents, and $df(w)$ is the number of documents in which
the term $w$ appears. $k_1 = 1.2$, $k_3 = 500$, and $b = 0.75$ are
parameters which were set to heuristic defaults. This function has been
shown to be very robust, capturing a large number of important information
retrieval heuristics~\cite{Fang:2005:Axiomatic}, motivating our choice.

Our selection features were precision and recall of the
selected lab data in the student case analysis when compared against the
instructor's assignment.

We investigate the predictive capacity of these features by exploring the
improvement of the model when predicting our most challenging labels
(``quality'' and ``clarity''). We ran ten separate experiments with
different randomized training sets consisting of 50\% of the data when
using different feature combinations to represent the student submissions.
We again report the average and standard deviation of the MAE on the test
set across the ten runs. To further explore whether the regression method
is truly capturing patterns relevant for grading, we compare its MAE
against the MAE obtained by using a na\"ive baseline: compute the most
frequent label in the training data, and then assign this label to all
examples in the test set.  Intuitively, this is a very reasonable baseline
when comparing MAE---if the labels are normally distributed, picking the
most frequent one will ensure an absolute error of zero for the majority of
the examples---while simultaneously being unhelpful for discriminative
grading (which the regression method hopes to capture). Our results are
summarized in
Tables~\ref{table:feature-comb}~and~\ref{table:feature-comb-clar}.

\begin{table}
    \begin{tabular}{r|l|l}
        & \textbf{Baseline} & \textbf{SVOR}\\\hline

        \textbf{sim} (3)
        & 0.9358 $\pm$ 0.0882
        & \textbf{0.8811 $\pm$ 0.0940}
        \\\hline

        \textbf{sim + sel} (5)
        & 0.9566 $\pm$ 0.1677
        & \textbf{0.8642 $\pm$ 0.0325}
        \\\hline

        \textbf{toks} (2646)
        & 0.9075 $\pm$ 0.0789
        & \textbf{0.7660 $\pm$ 0.0910$^\dagger$}
        \\\hline

        \textbf{all} (2651)
        & 0.9792 $\pm$ 0.1568
        & \textbf{0.7566 $\pm$ 0.0738$^\dagger$}
        \\\hline
    \end{tabular}
    \\\\
    {\footnotesize
    $\dagger$: statistically signifigant using an unpaired $t$-test with $p
    \leq 0.05$.}

    \caption{Effectiveness (in terms of MAE) of incorporating additional
    features in grade prediction for ``quality'' dimensions using SVOR
    methods compared to the mode-assigning baseline. Number of features is
    given in parenthesis.}
    \label{table:feature-comb}
\end{table}

\begin{table}
    \begin{tabular}{r|l|l}
        & \textbf{Baseline} & \textbf{SVOR}\\\hline

        \textbf{sim} (3)
        & 0.7906 $\pm$ 0.0771
        & \textbf{0.7830 $\pm$ 0.0836}
        \\\hline

        \textbf{sim + sel} (5)
        & \textbf{0.7623 $\pm$ 0.0649}
        & 0.7811 $\pm$ 0.0561
        \\\hline

        \textbf{toks} (2646)
        & 0.7528 $\pm$ 0.0550
        & \textbf{0.7415 $\pm$ 0.0597}
        \\\hline

        \textbf{all} (2651)
        & \textbf{0.7189 $\pm$ 0.0617}
        & 0.7226 $\pm$ 0.0527
        \\\hline
    \end{tabular}

    \caption{Similar experiment to Table~\ref{table:feature-comb}, but for
    ``clarity'' dimension.}
    \label{table:feature-comb-clar}
\vskip-10pt
\end{table}


We see that for the ``quality'' dimension, the model is able to
successfully learn generalizable patterns in our features to predict the
label with errors that are statistically significantly less than the
baseline method. In general, the token features dominate the performance,
but it would seem as though the similarity and selection features have
lower variability in the MAE. Again, this result suggests that there are
likely gains to be had by utilizing a more sophisticated feature selection
method to remove some of the noise introduced by extraneous token features.

However, the ``clarity'' label shows us that the problem is far from being
solved in a general sense. Here, we see that our method consistently fails
to beat the baseline method, with the winning method being seemingly
random. This indicates that the features we have selected thus far are more
tailored toward discrimination along certain dimensions of the grading
rubric than others. More work must be placed into developing features that
truly capture the ``clarity'' dimension to allow the model to extract the
patterns the instructor observes when grading along this dimension.

What this demonstrates is that automatic grading of complex assignments is
currently feasible, but perhaps only in a limited fashion. Careful feature
generation is required, but in some cases a model can be learned to
effectively grade assignments. We suspect that significant gains in
grading performance can be obtained in other dimensions through careful
feature generation.

\subsection{An Active Learning to Rank Approach}
Supervised learning, however, is not a particularly good fit for grading
practices in reality for a number of reasons. First, as we observed in
Tables~\ref{table:feature-comb}~and~\ref{table:feature-comb-clar}, outright
prediction of an ordinal grade can be very challenging due to the highly
concentrated nature of the dataset labels (see
Table~\ref{table:grade-stats}). The vast majority of grade information
available for the grade prediction task is centered around the mean,
leaving very little information in the tails for a supervised learner to
extract patterns from. (In some cases, for example, there are as few as one
example for the highest and lowest ordinal grade values). The result is
noisy output that may be inappropriate for using directly. However, it is
worth noting that ordinal grade prediction is a hard problem, even for
humans: a previous study suggests disagreement rates around $44\%$ for
short answer grading~\cite{Mohler:2009:EACL}. We suspect that this only
becomes larger as assignments become more complex and difficult to grade,
which makes the task of outright label prediction much more difficult for
the machine as well.

Perhaps a more reasonable approach, then, is to produce a ranked list of
assignments from best to worst. Annotators are typically more consistent at
providing judgments of the form ``is $a$ better than $b$?''\ than ``on a
scale from 1--5, how good is $a$?''~\cite{Callison-Burch:2007:WMT}, so it
is reasonable to suspect that a machine learning model could achieve better
results when trained using such pairwise judgments. If a system can provide
a good ranking of assignments, an instructor simply needs to assign
``cutoff'' points in this ranking to determine grades. This simplifies the
learning problem from attempting to predict an ordinal label for a specific
assignment to assigning a ranking to a set of assignments. This is a well
studied area in information retrieval called ``learning to
rank''~\cite{Joachims:2002:KDD}, and there are a wide variety of methods
available that one can use to learn a ranking function for documents given
a set of available features.

Second, a serious drawback of the supervised learning approach is the lack
of early feedback in the instructor--machine collaboration. In the
supervised approach, a model is learned on some subset of the assignments
to predict the grades of the remaining assignments, but the process of
selecting a subset of assignments for grading is not guided. It is possible
that randomly selecting assignments to grade results in mostly redundant
information being given to the grading model. Instead, a better approach
could be to employ active learning to allow the machine learning model to
guide the instructor in providing the supervision to make the most
effective use of his/her effort.

Building on these two observations, we propose the following ``pairwise
active learning to rank'' model for automatic grading, which will employ
the following process:
\begin{enumerate}
  \item Ask the instructor for comparative judgments on $k_1$ pairs of
    assignments.
  \item Learn a model using a learning-to-rank approach on the available
    pairwise judgments.\label{al:learn-model-step}
  \item Apply the model to all remaining unjudged pairs.
  \item Select an unjudged pair to present to the instructor for judgment.
    \label{al:select-training-step}
  \item Go to step~\ref{al:learn-model-step}.
\end{enumerate}
Instantiations of this general approach will differ mainly in
steps~\ref{al:learn-model-step}~and~\ref{al:select-training-step}.

Before we explore the efficacy of such an approach, however, we must first
redefine some measure by which we can measure performance. Because the
system is no longer predicting a rating for each assignment, we cannot use
MAE as before.

\subsubsection{Evaluating Ranking-based Grading Systems}
Our goal is to produce a ranking of student assignments that is consistent
with instructor evaluation. One way of framing this problem is to compare
the ranking produced by the system to the ranking produced by the
instructor (which we'll call the ``reference ranking''). A system's ranking
can then be evaluated using some measure of correlation between the two
rankings. We note a preference for metrics that take into account the
\emph{entire} ranked list---this contrasts with most of the preferred
measures in information retrieval evaluation which typically place heavier
emphasis on the top-ranked elements. While this makes sense in a search
context, our goal is to produce an exhaustive ranking of the assignments,
so we focus on these types of measures.

Measures for rank correlation are plentiful. Perhaps the most commonly
used metrics are Kendall's $\tau$ or Spearman's $\rho$ (which have been
found to be highly correlated in practice~\cite{Shani:2011:Springer}; thus,
we present only one for illustration). Kendall's $\tau$ can be formulated
as
\[
    \tau = \frac{n_c - n_d}{\frac{1}{2} n (n-1)},
\]
where $n_c$ is the number of \emph{concordant pairs}, and $n_d$ is the number
of \emph{discordant pairs}, and $n$ is the number of items ranked. To
compute $n_c$ and $n_d$, one considers all pairs $(x_i, y_i)$ and $(x_j,
y_j)$ (that is, pairs of tuples) of assigned rankings in the system ranking
$X$ and the reference ranking $Y$ (the denominator is simply the number of
such pairs). A pair is \emph{concordant} if the ordering of the items $i$
and $j$ in $X$ and $Y$ is consistent---in other words, if $(x_i < x_j)
\land (y_i < y_j)$ or $(x_i > x_j) \land (y_i > y_j)$.  A pair is
\emph{discordant} if the ordering of items in the two lists is
inconsistent---in other words, if $(x_i < x_j) \land (y_i > y_j)$ or $(x_i
> x_j) \land (y_i < y_j)$. This is then a correlation measure, with values
bounded in $[-1, 1]$, with $1$ indicating a perfect correlation and $-1$
indicating inverse correlation.

One of the assumptions Kendall's $\tau$ makes is that there are no ties in
ranks. However, in a realistic grading scenario based on rubrics we expect
many ties. Fortunately, there is a variation of Kendall's $\tau$, denoted
as $\tau_b$, that accounts for ties in the rankings. This is formulated as
\[
    \tau_b = \frac{n_c - n_d}{\sqrt{(n_c + n_d + t_x)(n_c + n_d + t_y)}}
\]
where $t_x$ is the number of pairs that were tied on \emph{only} their
ranking from $X$, and $t_y$ is the number of pairs that were tied on
\emph{only} their ranking from $Y$.

This may, at first glance then, seem like a good measure to use, but it is
not without its problems. Despite taking into account ties in the rankings,
it may still penalize a system for re-ordering items that were tied in the
reference ranking---in other words, we may be penalized for not correctly
identifying elements who are tied in the reference ranking. Consider a
simple example: suppose the ranking proposed by a system is $X =
(1,2,3,4,5,6)$ but the reference ranking is $Y = (1,1,2,2,3,4)$.
Intuitively, the system made no real mistakes in that no pair where the
reference ranking asserted an is in the wrong order in $X$. However, we'll
see that $\tau_b \approx 0.9309$, indicating that the system did not achieve
perfect correlation.

To address this issue, Yao~\cite{Yao:1995:JASIS} proposed the normalized
distance-based performance measure (NDPM), which computes a distance
between two rankings that is insensitive to a system's reordering of tied
elements in the reference ranking. NDPM is computed as
\[
    NDPM = \frac{2n_d + t_x}{2(n_c + n_d + t_x)}.
\]
Note that this is a \emph{distance} measure, so a value of $0$ would
indicate a perfect ranking. Indeed, if we compute NDPM for the example
rankings above, we achieve this result. Thus, we feel that NDPM is perhaps
the most appropriate measure for evaluating automatic grading systems that
produce an ordering of assignments as their output.

\subsubsection{Efficiently Utilizing Human Judgments with Active Learning}
To study whether our proposed active learning approach better utilizes
human judgments during the grading process, we performed the following
experiment. We took our assignments and assigned each a ``composite
score'', computed as the average of their ordinal score for each of the six
rubric dimensions. Our task is then to learn a ranking that is consistent
with the ranking produced by these composite scores while simultaneously
\emph{minimizing instructor effort} in providing the necessary supervision.

We first transform the $n = 107$ assignments into $\frac{1}{2}n(n-1) =
5671$ assignment \emph{pairs} $(x_i, x_j)$ with corresponding labels
$y_{ij} \in \{+1, -1\}$ indicating whether $x_i$ should be ranked above or
below $x_j$ in the ranking. Ties were broken arbitrarily by assignment id.
The supervision given by the instructor is then to indicate a preference
for ranking $x_i$ relative to $x_j$.

Following the process laid out in the beginning of the section, we first
start with $k_1 = 10$ random pairs selected from the transformed data and
ask for labels from the instructor. We then learn the model, compute the
NDPM for the ranking produced by the model for all $n$ assignments, and
then ask for additional supervision by selecting the unlabeled assignment
pair whose distance from the decision boundary for the model is lowest
(this is a known, simple approach to uncertainty
sampling~\cite{Settles:2012}) and repeat the training/evaluation loop. Our
particular model choice was a linear SVM provided through the \textsc{MeTA}
toolkit.

We compare this active learning scenario with two different baselines. The
first is a simple na\"ive approach that ranks all student assignments by
their BM25 similarity with the instructor document as a query. Our other
baseline is the exact same process as above, but instead of selecting the
most uncertain pair in the unlabeled data we select one uniformly at
random. This will allow us to see whether the uncertainty sampling approach
is truly helping to guide the learning process to make more efficient
supervision choices or not.

\begin{figure}
  \begin{center}
  \begin{tikzpicture}
    \begin{axis}[xlabel={human-labeled pairs},ylabel={NDPM}, xmax=500,
      legend style={font=\small}]

      \addplot+[color=red, each nth point={10}, filter discard
          warning=false, mark=triangle, error bars/.cd, y dir=both, y explicit]
        table [x=training-size, y=AVG-NDPM, y error=std-deviation, col
            sep=comma] {data/random-learning.csv};

      \addplot+[color=blue, each nth point={10}, filter discard
          warning=false, mark=x, error bars/.cd, y dir=both, y explicit]
        table [x=training-size, y=AVG-NDPM, y error=std-deviation, col
          sep=comma] {data/active-learning.csv};

      \addplot[mark=none, color=green]
        coordinates {(0, 0.344637) (500, 0.344637)};

      \legend{Random learning, Active learning, BM25 similarity}
    \end{axis}
  \end{tikzpicture}
  \caption{A comparison between a no-learning solution, a randomized
  learning solution, and an active learning solution to the
  grading-as-ranking problem. Reported is the average NDPM (lower is
  better) over 5 runs, with error bars indicating one standard deviation.}
  \label{fig:active-learning}
  \end{center}
\end{figure}

Our results are summarized in Figure~\ref{fig:active-learning}. We can
immediately observe that the learning-based approaches significantly
outperform the na\"ive text-similarity ranking baseline, which should be
expected. More importantly is the difference between the active learning
method (blue line) and the random learning method (red line). We can see
that even at a small fraction of all of the assignment pairs, the active
learning approach is able to achieve better NDPM than simply learning at
random. This is consistent with our hypothesis that active learning as part
of an automatic grading system can make more effective use of an
instructor's time than a purely passive supervised approach.
