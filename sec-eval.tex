
\section{Automated Grading as Ranking Assignments}

\subsection{Motivation}

Some of the results from our feasibility study using ordinal regression raise
 the question whether framing the problem of automated grading as
ordinal regression is appropriate. Indeed, as we will discuss, it appears
to be more advantageous to frame the problem as one of ranking
the ungraded assignments, which a human grader can segment  into
any desired grade levels. 

Specifically, as we observed in
Tables~\ref{table:feature-comb}~and~\ref{table:feature-comb-clar}, outright
prediction of an ordinal grade can be very challenging due to the highly
concentrated nature of the dataset labels (see
Table~\ref{table:grade-stats}). The vast majority of grade information
available for the grade prediction task is centered around the mean,
leaving very little information in the tails for a supervised learner to
extract patterns from. (In some cases, for example, there are as few as one
example for the highest and lowest ordinal grade values). The result is
noisy output that may be inappropriate for using directly. However, it is
worth noting that ordinal grade prediction is a hard problem, even for
humans: a previous study suggests disagreement rates around $44\%$ for
short answer grading~\cite{Mohler:2009:EACL}. We suspect that this only
becomes larger as assignments become more complex and difficult to grade,
which makes the task of outright label prediction much more difficult for
the machine as well.

Thus  an alternative, and more reasonable approach may be to produce a ranked list of
assignments from best to worst. Annotators are typically more consistent at
providing judgments of the form ``is $a$ better than $b$?''\ than ``on a
scale from 1--5, how good is $a$?''~\cite{Callison-Burch:2007:WMT}, so it
is reasonable to suspect that a machine learning model could achieve better
results when trained using such pairwise judgments. If a system can provide
a good ranking of assignments, an instructor simply needs to assign
``cutoff'' points in this ranking to determine grades. This simplifies the
learning problem from attempting to predict an ordinal label for a specific
assignment to assigning a ranking to a set of assignments. This is a well
studied area in information retrieval called ``learning to
rank''~\cite{Joachims:2002:KDD, Liu:2009}, and there are a wide variety of methods
available that one can use to learn a ranking function for documents given
a set of available features.  

One particular method that we will explore is ......  Before we explore the efficacy of such an approach, however, we must first
redefine some measure by which we can measure performance. Because the
system is no longer predicting a rating for each assignment, we cannot use
MAE as before.

\subsection{Evaluating Ranking-based Grading Systems}
Our goal is to produce a ranking of student assignments that is consistent
with instructor evaluation. One way of framing this problem is to compare
the ranking produced by the system to the ranking produced by the
instructor (which we'll call the ``reference ranking''). A system's ranking
can then be evaluated using some measure of correlation between the two
rankings. We note a preference for metrics that take into account the
\emph{entire} ranked list---this contrasts with most of the preferred
measures in information retrieval evaluation which typically place heavier
emphasis on the top-ranked elements. While this makes sense in a search
context, our goal is to produce an exhaustive ranking of the assignments,
so we focus on these types of measures.

Measures for rank correlation are plentiful. Perhaps the most commonly
used metrics are Kendall's $\tau$ or Spearman's $\rho$ (which have been
found to be highly correlated in practice~\cite{Shani:2011:Springer}; thus,
we present only one for illustration). Kendall's $\tau$ can be formulated
as
\[
    \tau = \frac{n_c - n_d}{\frac{1}{2} n (n-1)},
\]
where $n_c$ is the number of \emph{concordant pairs}, and $n_d$ is the number
of \emph{discordant pairs}, and $n$ is the number of items ranked. To
compute $n_c$ and $n_d$, one considers all pairs $(x_i, y_i)$ and $(x_j,
y_j)$ (that is, pairs of tuples) of assigned rankings in the system ranking
$X$ and the reference ranking $Y$ (the denominator is simply the number of
such pairs). A pair is \emph{concordant} if the ordering of the items $i$
and $j$ in $X$ and $Y$ is consistent---in other words, if $(x_i < x_j)
\land (y_i < y_j)$ or $(x_i > x_j) \land (y_i > y_j)$.  A pair is
\emph{discordant} if the ordering of items in the two lists is
inconsistent---in other words, if $(x_i < x_j) \land (y_i > y_j)$ or $(x_i
> x_j) \land (y_i < y_j)$. This is then a correlation measure, with values
bounded in $[-1, 1]$, with $1$ indicating a perfect correlation and $-1$
indicating inverse correlation.

One of the assumptions Kendall's $\tau$ makes is that there are no ties in
ranks. However, in a realistic grading scenario based on rubrics we expect
many ties. Fortunately, there is a variation of Kendall's $\tau$, denoted
as $\tau_b$, that accounts for ties in the rankings. This is formulated as
\[
    \tau_b = \frac{n_c - n_d}{\sqrt{(n_c + n_d + t_x)(n_c + n_d + t_y)}}
\]
where $t_x$ is the number of pairs that were tied on \emph{only} their
ranking from $X$, and $t_y$ is the number of pairs that were tied on
\emph{only} their ranking from $Y$.

This may, at first glance then, seem like a good measure to use, but it is
not without its problems. Despite taking into account ties in the rankings,
it may still penalize a system for re-ordering items that were tied in the
reference ranking---in other words, we may be penalized for not correctly
identifying elements who are tied in the reference ranking. Consider a
simple example: suppose the ranking proposed by a system is $X =
(1,2,3,4,5,6)$ but the reference ranking is $Y = (1,1,2,2,3,4)$.
Intuitively, the system made no real mistakes in that no pair where the
reference ranking asserted an is in the wrong order in $X$. However, we'll
see that $\tau_b \approx 0.9309$, indicating that the system did not achieve
perfect correlation.

To address this issue, Yao~\cite{Yao:1995:JASIS} proposed the normalized
distance-based performance measure (NDPM), which computes a distance
between two rankings that is insensitive to a system's reordering of tied
elements in the reference ranking. NDPM is computed as
\[
    NDPM = \frac{2n_d + t_x}{2(n_c + n_d + t_x)}.
\]
Note that this is a \emph{distance} measure, so a value of $0$ would
indicate a perfect ranking. Indeed, if we compute NDPM for the example
rankings above, we achieve this result. Thus, we feel that NDPM is perhaps
the most appropriate measure for evaluating automatic grading systems that
produce an ordering of assignments as their output.

\section{Efficiently Utilizing Human Judgments with Active Learning}
As in all supervised learning approaches, 
the accuracy of  the automated grader based on learning to rank. depends
on the quantity and quality of the training examples available for the 
learner to use. Ideally, we would like human graders to provide
 as many graded examples as possible, but this would
reduce the benefit of an automated grader. Indeed,
if a human grader completes grading all the assignments,
there would be no need for the automated grader!
However, if there are insufficient training examples 
to learn from, the automated grader migh have a low accuracy,
which would further require more human effort on 
``post-editing" the grading results of the automated grader. 
Thus there is clearly a complicated tradeoff between
the effort of manual grading and the utility of the trained
 grader that may have to be empirically optimized in an application-specific way.

However, it is very clear that if we ask human graders to grade
a certain amount of assignments, we would like the
graded assignments to be as useful to the automated grader as possible.
Just randomly selecting a sample of assignments for manual grading
is not the best way. 
\ignore{
A serious drawback of the supervised learning approach is the lack
of early feedback in the instructor--machine collaboration. In the
supervised approach, a model is learned on some subset of the assignments
to predict the grades of the remaining assignments, but the process of
selecting a subset of assignments for grading is not guided. It is possible
that randomly selecting assignments to grade results in mostly redundant
information being given to the grading model. Instead, }
A natural solution to this problem is to employ 
active learning to allow the machine learning model to
guide the instructor in providing the supervision to make the most
effective use of his/her effort.

Building on these observations, we thus propose the following ``pairwise active
learning to rank'' model for automatic grading, which will employ the
following process where $k_1$ is a parameter that can be empirically set:
\begin{enumerate*}[label=\itshape(\arabic*)\upshape]
  \item Ask the instructor for comparative judgments on $k_1$ pairs of
    assignments,

  \item Learn a model using a learning-to-rank approach on the available
    pairwise judgments\label{al:learn-model-step},

  \item Apply the model to all remaining unjudged pairs,

  \item Select an unjudged pair to present to the instructor for
    judgment\label{al:select-training-step}, and

  \item Go to step~\ref{al:learn-model-step}.
\end{enumerate*}
Instantiations of this general approach will differ mainly in
steps~\ref{al:learn-model-step}~and~\ref{al:select-training-step}.

% subsection heading removed for space...
%\subsection{Experiments}

To study whether our proposed active learning approach better utilizes
human judgments during the grading process, we performed the following
experiment. We took our assignments and assigned each a ``composite
score'', computed as the average of their ordinal score for each of the six
rubric dimensions. Our task is then to learn a ranking that is consistent
with the ranking produced by these composite scores while simultaneously
\emph{minimizing instructor effort} in labeling.

We first transform the $n = 107$ assignments into $\frac{1}{2}n(n-1) =
5671$ assignment \emph{pairs} $(x_i, x_j)$ with corresponding labels
$y_{ij} \in \{+1, -1\}$ indicating whether $x_i$ should be ranked above or
below $x_j$ in the ranking. Ties were broken arbitrarily by assignment id.
The supervision given by the instructor is then to indicate a preference
for ranking $x_i$ relative to $x_j$.

Following the process laid out in the beginning of the section, we first
start with $k_1 = 10$ random pairs selected from the transformed data and
ask for labels from the instructor. We then learn the model, compute the
NDPM for the ranking produced by the model for all $n$ assignments, and
then ask for additional supervision by selecting the unlabeled assignment
pair whose distance from the decision boundary for the model is lowest
(this is a known, simple approach to uncertainty
sampling~\cite{Settles:2012}) and repeat the training/evaluation loop. Our
particular model choice was a linear SVM provided through the \textsc{MeTA}
toolkit.

We compare this active learning scenario with a random learning baseline,
which is the exact same process as above, but instead of selecting the most
uncertain pair in the unlabeled data we select one uniformly at random.
This will allow us to see whether the uncertainty sampling approach is
truly helping to guide the learning process to make more efficient
supervision choices or not.

\begin{figure}
  \begin{center}
   \begin{tikzpicture}[scale=0.9]
    \begin{axis}[xlabel={human-labeled pairs},ylabel={NDPM}, xmax=500,
      legend style={font=\small}]

      \addplot+[color=red, each nth point={10}, filter discard
          warning=false, mark=triangle, error bars/.cd, y dir=both, y explicit]
        table [x=training-size, y=AVG-NDPM, y error=std-deviation, col
            sep=comma] {data/random-learning.csv};

      \addplot+[color=blue, each nth point={10}, filter discard
          warning=false, mark=x, error bars/.cd, y dir=both, y explicit]
        table [x=training-size, y=AVG-NDPM, y error=std-deviation, col
          sep=comma] {data/active-learning.csv};

%      \addplot[mark=none, color=green]
%        coordinates {(0, 0.344637) (500, 0.344637)};

      \legend{Random learning, Active learning, BM25 similarity}
    \end{axis}
  \end{tikzpicture}
  \caption{A comparison between a randomized learning solution and an
  active learning solution to the grading-as-ranking problem. Reported is
  the average NDPM (lower is better) over 5 runs, with error bars indicating
  one standard deviation.}
  \label{fig:active-learning}
\vskip-10pt
  \end{center}
\end{figure}

Our results are summarized in Figure~\ref{fig:active-learning}. Recall that
a NDPM value of $0.3$ indicates that a system ranking was 30\% of the
maximal achievable distance away from the reference ranking. We can see
that even at a small fraction of all of the assignment pairs, the active
learning approach (blue line) is able to achieve better NDPM than simply
learning at random (red line). This is consistent with our hypothesis that
active learning as part of an automatic grading system can make more
effective use of an instructor's time than a purely passive supervised
approach.

How much instructor effort goes in to judging 200 assignment pairs? This
may initially seem like a lot, but each pair is not labeled in
isolation---labeling many pairs will inevitably include assignments that
have already been seen before. These familiar assignments make providing a
pairwise judgment faster than it would be if done ``cold''. In general, it
is also reasonable to assume that the effort involved in simply saying
whether assignment $a$ is better than assignment $b$ is lower than having
to consult a rubric to assign an actual point (or letter) value. It is
important, however, to ensure that providing a pairwise judgment takes as
little effort as possible relative to assigning a numeric or letter grade.
An interesting future direction is then to design an interactive system
that attempts to further drive the cost of providing judgments down.

